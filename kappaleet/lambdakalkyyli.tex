Funktionaalisen ohjelmoinin teoreettisena perustana on Alonzo Churchin kehittämä lambdakalkyyli. Lambdakalkyyli on matemaattinen laskennan malli, joka kuvaa laskennan joukkona sievennyssääntöjä, jotka operoivat lambdalausekkeilla. Lambdalausekkeet voidaan määritellä esimerkiksi seuraavasti:


\begin{maar}[Lambdalausekkeet]
Määritellään lambdalausekkeiden joukko $E$ rekursiivisesti: 
\[ Id \subset E \]
\[ \text{Jos } e_{1} \in E \text{ ja } e_{2} \in E, \text{ niin }  (e_{1} \; e_{2}) \in E \]
\[ \text{Jos } x \in Id \text{ ja } e \in E, \text{ niin } \lambda x.e \in E \]

missä $Id$ on numeroituvasti ääretön joukko muuttujia. Lauseketta joka on muotoa $\lambda x.e$ kutsutaan lambda-abstraktioksi tai lyhyesti abstraktioksi. Lauseketta, joka on muotoa $(e_{1} \; e_{2})$ kutsutaan lausekkeen $e_{2}$ aplikaatioksi lauseekkeeseen $e_{1}$ tai lyhyesti aplikaatioksi. Käytämme aplikaatiolle vasemmalle assosioivaa notaatiota, esimerkiksi: 
$(e_{1} \; e_{2} \; e_{3}) = ((e_{1} \; e_{2}) \; e_{3})$. Abstraktiolle käytetään seuraavaa lyhennysmerkintää: 
$(\lambda x_{1} . ( \lambda x_{2} . ( \: ... \: ( \lambda x_{n} . e ))) \: ... \: ) = \lambda x_{1}x_{2}...x_{n}.e$
\end{maar}

\begin{esim}
Esimerkiksi seuraavat lausekkeet ovat lambdalausekkeita

\[ z \]
\[ \lambda xy . ( y \; x \; z ) \]
\[ \lambda x . ( x \; \lambda y . z ) \]
 
\end{esim}

