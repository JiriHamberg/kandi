\section{Johdanto}

Imperatiivisessa ohjelmoinnissa ohjelma esitetään sarjana komentoja, jotka muokkaavat ohjelman tilaa (\textit{state}). Ohjelman tilalla tarkoitetaan ohjelman muuttuja-arvo -parien kokoelmaa. Imperatiivisen ohjelmointiparadigman teoreettisena perustana on laskennan malli, joka tunnetaan Turingin koneena. Turingin kone koostuu nauhasta ja lukupäästä sekä tilasiirtymäfunktiosta. Nauhan sisältö ja lukupään sijainti nauhalla muodostavat yhdessä Turingin koneen tilan, joka voidaan samastaa imperatiivisen ohjelman tilan kanssa. Siirtymäfunktio voidaan puolestaan samastaa imperatiiviseen ohjelmakoodiin. Turingin kone esittää laskennan sarjana nauhan ja lukupään tilanmuutoksia, jotka siirtymäfunktio määrittelee yksikäsitteisesti. Juuri laskennan esittäminen sarjana tilanmuutoksia karakterisoi parhaiten imperatiivisen ohjelmointiparadigman~\cite[s.~3]{Hudak89}.
\par
Funktionaalisessa ohjelmoinnissa ohjelman esitys on yksi tai useampi lauseke. Ohjelman suorittaminen merkitsee lausekkeiden sieventämistä normaalimuotoon reduktiosääntöjä käyttäen. Huomattavin ero funktionaalisen ja imperatiivisen paradigman välillä on se, että funktionaaliset ohjelmat ovat tilattomia. Eräs tilattomuuden suurista eduista suhteessa imperatiiviseen ohjelmointityyliin on viittausläpinäkyvyys (\textit{referential transparency}). Lausekkeen viittausläpinäkyvyys tarkoittaa sitä, että lausekkeen mikä tahansa alilauseke voidaan korvata toisella lausekkeella, jonka arvo on sama kuin alilausekkeen arvo, muuttamatta ohjelman semantiikkaa~\cite[s.~5]{Hudak89}.  
\par
Funktionaalisen ohjelmoinnin teoreettisena perustana on Alonzo Churchin 1930-luvulla kehittämä lambdakalkyyli ~\cite[s.~37--50]{PJ1987}. Lambdakalkyyli on laskennan malli, jossa laskennalla tarkoitetaan annetun lambdalausekkeen sieventämistä normaalimuotoon käyttäen lambdalausekkeille määriteltyjä reduktiosääntöjä.