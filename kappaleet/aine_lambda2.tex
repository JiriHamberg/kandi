\section{Reduktiosäännöt ja yhtäsuuruus}

\subsection{Reduktiosäännöt}
Lambdalausekkeiden kiinnostavimpia piirteitä lienee, että niille voidaan muotoilla varsin yksinkertaiset laskusäännöt, jotka kuitenkin riittävät tekemään lambdakalkyylista Turing-täydellisen~\cite[liite]{Turing36}.  
\par
Kuten jo edellä mainittiin, voidaan lambda-abstraktio samastaa anonyymin funktion kanssa. On luontevaa ajatella, että lambdalausekkeen $M$ sovellus lambda-abstraktiolle $\lambda x . N$ vastaa abstraktin funktion $\lambda x . N$ arvoa argumentilla $M$. Tällöin reduktiosäännöt voidaan ymmärtää näiden anonyymien funktioden sievennyssäännöiksi.
\par
Määritellään seuraavaksi lambdalausekkeiden reduktiosäännöt~\cite[s.~9]{Hudak89}. Reduktiosäännöt kertovat miten lambdalausekkeita voidaan sieventää muuttamatta niiden arvoa. Beta-reduktio ilmaisee miten lambda-abstraktiosta ja mielivaltaisesta lambdalausekkeesta koostuvaa sovellusta voidaan sieventää. Eta-reduktiota puolestaan ilmaisee sen, että jos lambda-abstraktio ei tee muuta kuin soveltaa muuttujaansa lambdalausekkeeseen $e$, niin abstraktion arvo on sama kuin lausekkeen $e$.    

\begin{maar}[reduktiosäännöt]	
	%\begin{enumerate}		
\[\beta \text{-reduktio: } \; (\lambda x.e )\; e_{1} \rightarrow_{\beta} [e_{1} / x]e \]
\[\eta \text{-reduktios: } \; \lambda x.(e \; x) \rightarrow_{\eta} e \text{ jos } x \notin fv(e) \]	
	%\end{enumerate}
Jos $\beta$-reduktion toistuva soveltaminen lausekkeen $N$ alilausekkeisiin tuottaa lausekkeen $M$, niin sanotaan, että $M$ on lausekkeen $N$ \textbf{beta-reduktio} ja merkitään $N  \twoheadrightarrow_{\beta} M$. Vastaavasti Jos $\eta$-reduktion toistuva soveltaminen lausekkeen $N$ alilausekkeisiin tuottaa lausekkeen $M$, niin sanotaan, että $M$ on lausekkeen $N$ \textbf{eta-reduktio} ja merkitään $N  \twoheadrightarrow_{\eta} M$. Lisäksi, jos lauseke $M$ saadaan lausekkeesta $N$ soveltamalla sen alilausekkeisiin toistuvasti eta- tai beta-reduktiota, niin sanotaan, että $M$ on lausekkeen $N$ \textbf{reduktio} ja merkitään $N  \twoheadrightarrow M$        
\end{maar}

%\begin{esim}[$\beta$-reduktio]
%<ESIMERKKI>
%\end{esim}
%
%\par

\subsection{Lausekkeiden yhtäsuuruus}

Seuraavaksi esiteltävät relaatiot~\cite[s.~23--24]{HBEB2000} karakterisoivat lambdalausekkeiden arvojen yhtäsuuruuden ja niiden ideana on toimia yksinkertaisina sääntöinä, joiden avulla voidaan päätellä ovatko kaksi lambdalauseketta arvoiltaan yhtäsuuret.
\par
Intuitiivisesti on selvää, että jos kaksi lauseketta voidaan sieventää samaan muotoon, niin lausekkeilla on sama arvo. Lausekkeiden yhtäsuuruus määritelläänkin pitkälti edellä annettujen sievennyssääntöjen avulla. Lisäksi halutaan, että kaksi lambdalauseketta ovat yhtä suuret, jos ne eroavat toisistaan vain sidottujen muuttujiensa nimien suhteen.  
\par
Kaksi lambdalauseketta ovat alfa-ekvivalentit, mikäli ne ne eroavat toisistaan ainoastaan sidottujen muuttijiensa nimien suhteen. Kaksi lauseketta ovat beta- tai eta-ekvivalentit jos toinen niistä saadaan toisesta käyttämällä beta- tai eta-reduktiota.

\begin{maar}[lambdalausekkeiden arvojen yhtäsuuruus]
	%\begin{enumerate}
		 
\[ 
	\alpha \text{-ekvivalenssi: } \; \lambda x.e =_{\alpha} \lambda x_{i}.		[x/x_{i}]e \text{,  } \forall x_{i} \notin fv(e) 
\]
\[ 
	\beta \text{-ekvivalenssi: } \; N =_{\beta} M \text{ jos ja vain jos } N 		\twoheadrightarrow_{\beta} M \text{ tai }
	M \twoheadrightarrow_{\beta} N 
\]
\[ 
	\eta \text{-ekvivalenssi: } \; N =_{\eta} M \text{ jos ja vain jos } N 		\twoheadrightarrow_{\eta} M \text{ tai }
	M \twoheadrightarrow_{\eta} N 
\]
	%\end{enumerate}
Lambdalausekkeet $N$ ja $M$ ovat arvoltaan yhtäsuuret, mikäli	on olemassa lambdalauseke $L$ siten että $N$ ja $L$ ovat keskenään alfa- beta- tai eta-ekvivalenttit ja $M$ ja $L$ ovat keskenään alfa-, beta- tai eta-ekvivalentit.	
\end{maar}

Intuitiivisesti edellinen määritelmä tarkoittaa, että kaksi lambdalauseketta ovat arvoiltaan yhtäsuuret, mikäli on olemassa lauseke, joka voidaan johtaa molemmista lausekkeista käyttäen eta- ja beta-reduktiota sekä uudelleennimeämällä sidottuja muuttujia. 

%\par

 %Esimerkiksi seuraavat lambdalausekkeet ovat arvoiltaan yhtäsuuret:
 
%Lambdalausekkeen sanotaan olevan normaalimuodossa, jos siihen ei voida soveltaa $\eta$- tai $\beta$-reduktioita. Voidaan osoittaa, että kaikille lambdalausekkeille, jotka voidaan redusoida normaalimuotoon, pätee, että normaalimuoto saavutetaan valitsemalla aina lausekkeen vasemmanpuolisin alilauseke, jota voidaan vielä redusoida ja soveltamalla siihen $\eta$- tai $\beta$-reduktiota ja toistamalla tätä menettylyä kunnes lauseke on normaalimuodossa. Lisäksi voidaan osoittaa että jos lauseke voidaan redusoida normaalimuotoon, niin lausekkeen normaalimuoto on yksikäsitteinen. Näinollen lambdalausekkeen arvon laskemisella tarkoitetaan lausekkeen redusoimista normaalimuotoon.
%\par
%Kaikille lambdalausekkeille ei ole olemassa normaalimuotoa. Alan Turing todisti vuonna 1937, että Turingin kone on laskennan mallina ekvivalentti lambdakalkyylin kanssa. Tämä tulos ei voisi pitää paikkaansa, jos jokaiselle lambdalausekkeelle olisi löydettävissä normaalimuoto. Muuten ratkeamattomaksi tunnettu pysähtymisongelma voitaisiin ratkaista simuloimalla Turingin koneita lambdalausekkeilla.
%\par 
%Vaikka lambdalausekkeet ovat rakenteeltaan melko yksinkertaisia, voidaan niiden avulla ymmärtää ja mallintaa monia funktionaalisen ohjelmoinnin kannalta tärkeitä käsitteitä. Koska lambda-abstraktiot voivat ottaa parametrinaan ja antaa paluuarvonaan lambdafunktion, ovat esimerkiksi korkeamman asteen funktiot varsin luonnollinen osa lambdakalkyylia.
