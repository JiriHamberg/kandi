\section{Normaalimuoto ja evaluointi}

Edellä määriteltiin, millaiset muunnokset säilyttävät lambdalausekkeen arvon samana. Jotta lambdalausekkeilla voidaan suorittaa laskentaa, täytyy lausekkeen arvo ja lauseke kyetä erottamaan toisistaan. Laskennan määritelmä tulisi muotoilla siten, että (pseudokoodia):

\[ 
(\lambda x. 3 * x + 5) 6 \; \rightarrow \; 3 * 6 + 5 \; \rightarrow \; 18 + 5 \; \rightarrow \; 23 
\]

mutta 

\[ 18 + 5 \; \cancel{\rightarrow} \; (\lambda x. 3 * x + 5) 6 \]

toisinsanottuna laskennalla tulisi olla tietty suunta ja laskennalla tulisi olla yksikäsitteinen päätepiste aina kun laskenta päättyy. Lambdalausekkeeseen laskennan päätepistettä kutsutaan lausekkeen normaalimuodoksi ja se määritellään seuraavasti: 

\begin{maar}[normaalimuoto]
Lambdalauseke $L$ on normaalimuodossa, jos mihinkään sen alilausekkeista ei voida soveltaa $\beta$- tai $eta$-reduktiota ja jonka sidotuttujen muuttujien nimet ovat aakkosjärjestyksessä vasemmalta oikealle ja lausekkeen oikeanpuolisin sidottu muuttuja on aakkosjärjestyksessä pienin mahdollinen.  
\end{maar}

Yleensä sidottujen muuttujien uudelleennimeämistä ei ole tapana kirjoittaa auki, vaan ajatellaan, että lauseke on normaalimuodossa, kunhan sen alilausekkeisiin ei voida soveltaa $\beta$- tai $\eta$-reduktiota. Tällöin sidottujen muuttujien uudelleennimeäminen normaalimuodon vaatimalla tavalla on varsin suoraviivaista käyttäen $\alpha$-muunnosta.

\par

Lambdalausekkeiden evaluointi on siis lausekkeen johtamista normaalimuotoon, käyttäen $\beta$- ja $\eta$-reduktioita ja $\alpha$-muunnosta (sidottujen muuttujien uudelleennimeäminen) sen alilausekkeisiin. Erilaisia tapoja tämän tekemiseksi kutsutaan evaluaatiostrategioiksi. Tyypillisiä evaluaatiostrategiota ovat sovellusjärjestysevaluaatio (\textit{applicative-order evaluation}) ja normaalijärjestysevaluaatio (\textit{normal-order evaluation)}.

\begin{alg}[sovellusjärjestysevaluaatio] 
Lambdalausekkeen $L$ sovellusjärjestysevaluaatio on evaluaatiostrategia, jossa redusoidaan aina lausekkeen $L$ sisimmäista vasemmanpuolisinta alilauseketta, joka ei ole normaalimuodossa. (Sisimmäinen alilauseke on se alilauseke, jota ympäröi suurin määrä sulku-merkkejä.)
\end{alg}

\begin{alg}[normaalijärjestysevaluaatio]
Lambdalausekkeen $L$ normaalijärjestysevaluaatio on evaluaatiostrategia, jossa redusoidaan aina lausekkeen $L$ vasemmanpuolisinta alilauseketta, joka ei ole normaalimuodossa.
\end{alg}

\begin{esim}
Tarkastellaan, mitä tapahtuu lausekkeelle $(\lambda x . y) ((\lambda x . xx) (\lambda x . xx))$, kun siihen sovellataan eri evaluaatiostrategioita. Huomataan aluksi, että  

\[ (\lambda x . xx) (\lambda x . xx) \rightarrow_{\beta} (\lambda x . xx) (\lambda x . xx) \]

siten normaalijärjestysevaluaatiolla:

\[ (\lambda x . y) ((\lambda x . xx) (\lambda x . xx)) \Longrightarrow y \]

mutta sovellusjärjestysevaluaatiolla:

\[  (\lambda x . y) ((\lambda x . xx) (\lambda x . xx)) \Longrightarrow
	(\lambda x . y) ((\lambda x . xx) (\lambda x . xx)) \Longrightarrow
	(\lambda x . y) ((\lambda x . xx) (\lambda x . xx)) \Longrightarrow \ldots \]
	
Sovellusjärjestysevaluaatio ei siis tässä tapauksessa koskaan löydä lausekkeen normaalimuotoa. Tämän esimerkin perusteella evaluaatiostrategian valinnalla on merkitystä sen kannalta päättyykö laskenta vai ei. 
\end{esim}


\begin{lau}[Churchin-Rosserin teoreemat]
\begin{enumerate} $ $\newline
	\item Kullakin lambdalausekkeella on korkeintaan yksi normaalimuoto.
	\item Jos lausekkeella on normaalimuoto, niin normaalijärjestysevaluaatio löytää sen. 
\end{enumerate}
\end{lau}

Churchin-Rossierin teoreema 1. takaa, että lambdalausekkeiden laskenta on ristiriidatonta: mikään laskenta ei voi johtaa kahteen eri lopputulokseen. Toisaalta teoreema 2. takaa, että mikä tahansa lauseke, joka on mahdollista evaluoida, voidaan evaluoida käyttämällä normaalijärjestysevaluaatiota.

\par

Tähän mennessä olemme käsitelleet lambdalausekkeiden loogisia ja syntaktisia piirteitä liittämättä lausekkeisiin minkäänlaista semantiikkaa. Näytetään seuraavaksi miten lambdalausekkeilla voidaan määritellä kokonaisluvut ja hieman kokonaislujen aritmetiikkaa.
   
\begin{esim}[Churchin numeraalit ja aritmetiikka]
Kokonaisluvut voidaan määritellä monin eri tavoin käyttäen lambdalausekkeita. Näistä tunnetuin lienee Churchin alkuperäinen muotoilu, jossa

\[ \textbf{0} \equiv \lambda fx . x \]
\[ \textbf{1} \equiv \lambda fx . fx \]
\[ \textbf{2} \equiv \lambda fx . f (fx) \] 
\[ \textbf{3} \equiv \lambda fx . f(f(fx)) \]
\[ \vdots \]
\[ \textbf{N} \equiv \lambda fx . \underbrace{ f ( f \ldots (f }_{ N-\text{kappaletta}} x)) \ldots ) \]
 
Kokonaisluku $N$ esitetään siis lausekkeena, jonka ensimmäistä parametria on sovellettu $N$-kertaa itselleen.

\par

Nyt yhteenlasku-funktio  \textbf{PLUS} voidaan määritellä seuraavasti:

\[ \textbf{ADD} \equiv \lambda nmfx . n f (m f x) \] 

sillä 

%\begin{align}
%\textbf{PLUS} N M &\equiv (\lambda fx . \underbrace{ f ( f \ldots (f }_{ N-\text{kappaletta}}) \\
%&= foo bar
%\end{align}

\begin{align*} \textbf{ADD N M} &\equiv (\lambda nmfx . n f (m f x)) \textbf{N M} \\ 
&\equiv  (\lambda nmfx . n f (m f x)) \; (\lambda fx . \underbrace{ f ( f \ldots (f }_{ N-\text{kappaletta}} x)) \ldots )) \; (\lambda fx . \underbrace{ f ( f \ldots (f }_{ M-\text{kappaletta}} x)) \ldots ))\\ 
&\rightarrow_{\beta} (\lambda mfx . ((\lambda fx . \underbrace{ f ( f \ldots (f }_{ N-\text{kappaletta}} x)) \ldots ) f (m f x)) \; \lambda fx . \underbrace{ f ( f \ldots (f }_{ M-\text{kappaletta}} x)) \ldots )\\ 
&\rightarrow_{\beta} \lambda fx . ((\lambda fx . \underbrace{ f ( f \ldots (f }_{ N-\text{kappaletta}} x)) \ldots ) f ( (\lambda fx . \underbrace{ f ( f \ldots (f }_{ M-\text{kappaletta}} x)) \ldots )) f x ) \\
&\rightarrow_{\beta} \lambda fx . ((\lambda x . \underbrace{ f ( f \ldots (f }_{ N-\text{kappaletta}} x)) \ldots ) ( (\lambda fx . \underbrace{ f ( f \ldots (f }_{ M-\text{kappaletta}} x)) \ldots )) f x ) \\
&\rightarrow_{\beta}  \lambda fx . \underbrace{ f ( f \ldots (f }_{ N-\text{kappaletta}} ( (\lambda fx . \underbrace{ f ( f \ldots (f }_{ M-\text{kappaletta}} x)) \ldots ) f x ) \ldots )\\
&\rightarrow_{\beta}  \lambda fx . \underbrace{ f ( f \ldots (f }_{ N-\text{kappaletta}} ( (\lambda x . \underbrace{ f ( f \ldots (f }_{ M-\text{kappaletta}} x)) \ldots )  x ) \ldots) \\
&\rightarrow_{\beta}  \lambda fx . \underbrace{ f ( f \ldots (f }_{ N-\text{kappaletta}} (\underbrace{ f ( f \ldots (f }_{ M-\text{kappaletta}} x) \ldots ) \ldots ) \\
&\equiv \lambda fx . \underbrace{ f ( f \ldots (f }_{ N + M-\text{kappaletta}} x) \ldots ) \\
&\equiv \textbf{N + M}  
\end{align*}

Kertolasku-funktiolle \textbf{MUL} voidaan antaa seuraava määritelmä:

\[ \textbf{MUL} \equiv \lambda n m f . n (mf) \]

Identiteetti $\textbf{MUL N M } \equiv \textbf{N} * \textbf{M}$ voidaan johtaa samaan tapaan kuin funktion \textbf{ADD} tapauksessa suoraviivaisesti (joskin vaivalloisesti) $\beta$-reduktiota käyttämällä. 

\par

Näytetään vielä miten lauseke $2 * (1+1)$ voidaan esittää ja evaluida käyttäen Churchin numeraaleja. Ensinnäkin käyttäen Puolalaista notaatiota, lausekkeelle saadaan seuraava muotoilu

\[* \; 3 \; (+ \; 1 \; 1) \], 

joka voidaan suoraviivaisesti muuntaa lambdalausekkeeksi

\[ \textbf{MUL 3 }(\textbf{ADD 1 1}) \]

ja 

\begin{align*}
\textbf{MUL 3 }(\textbf{ADD 1 1}) &\equiv (\lambda n m f . n (mf))  \; (\lambda f x . f(f(fx))) \;(\textbf{ADD 1 1}) \\
&\twoheadrightarrow_{\beta} (\lambda n m f . n (mf) ) \; (\lambda f x . f(f(fx))) \; \textbf{2} \\
&\rightarrow_{\beta} (\lambda m f . (\lambda f x . f(f(fx))) \; (m f)) \textbf{2}    \\
&\rightarrow_{\beta} \lambda f . (\lambda f x . f(f(fx))) (\textbf{2} \; f)) \\
&\equiv \lambda f x . (\lambda f . f(f(fx))) ((\lambda f x . f (f x) \; f) \\
&\rightarrow_{\beta} \lambda f . (\lambda f x .  f(f(fx)) (\lambda x . f (f x)) \\
&\rightarrow_{\beta} \lambda f . (\lambda x .  (\lambda x . f (f x))((\lambda x . f (f x))((\lambda x . f (f x))x))\\
&\rightarrow_{\beta} \lambda f . (\lambda x .  (\lambda x . f (f x))((f (f( f( f x))))  \\
&\rightarrow_{\beta} \lambda f . (\lambda x . f( f (f (f( f( f x)))))) \\
&\equiv \lambda f x. f( f (f (f( f( f x))))) \\
&\equiv \textbf{6}
\end{align*}
 
%&\rightarrow_{\beta} (\lambda m f x . (\lambda f x . f(fx)) (mfx) x) \; (\textbf{ADD 1 1}) \\
%&\rightarrow_{\beta} (\lambda f x . (\lambda f x .f(fx)) (\textbf{ADD 1 1}fx)) x \\
%&\rightarrow_{\beta}  \lambda f x . (\lambda x . (\textbf{ADD 1 1} \; f \; x)(\textbf{ADD 1 1} \; f \; %x)) x \\
%&\rightarrow_{\beta}  \lambda f x . (\textbf{ADD 1 1} \; f \; x)(\textbf{ADD 1 1} \; f \; x) \\
%&\twoheadrightarrow_{\beta} \lambda fx . 
 
 
\end{esim}   
   