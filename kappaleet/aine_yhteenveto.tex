\section{Yhteenveto}

Lambdalausekkeet muodostavat funktionaalisen ohjelmoinnin teoreettisen perustan. Lambdalausekkeilla voidaan varsin luonnollisesti esittää funktionaaliselle ohjelmoinnille tyypillisiä käsitteitä kuten Curryn-muunnos ja korkeamman kertaluvun funktiot. Lambdalausekkeiden laskenta on tilatonta ja sivuvaikutuksetonta. 

\par

Lambdalausekkeet voidaan esittää matemaattisen logiikan formaalina systeeminä. Lausekkeille voidaan määrittelä lekvivalenssirelaatiot sekä reduktiosäännöt käyttäen apuna vapaiden muuttujien ja substituutiosääntöjen käsitteitä. Reduktiosäännöt ja ekvivalenssirelaatiot voidaan ymmärtää lausekkeiden abstrakteina laskusääntöinä.

\par

Lambdalausekkeilla, joiden laskenta päättyy, on olemassa yksikäsitteinen normaalimuoto, joka voidaan samastaa lambdalausekkeen arvon kanssa. Algoritmi nimeltään normaalijärjestysevaluointi muuntaa lambdalausekkeen sen normaalimuotoon jos lausekkeella on normaalimuoto. Kokonaislukuja voidaan mallintaa lambdalausekkeiden avulla käyttäen Churchin numeraaleja. Churchin numeraaleille voidaan muodostaa esimerkiksi yhteen- ja kertolaskufunktiot käyttäen lambdalausekkeita. 

\par 

Tässä työssä ei ole käsitelty rekursiivisia lambdafunktioita, vaikka niihin liittyvät tulokset ovat sekä historiallisesti että lambdalausekkeiden sovellusten kannalta erittäin olennaisia. Muita kiinnostavia jatkokysymyksiä ovat lambdalausekkeiden sovellukset funktionaalisten kielten toteutuksissa, esimerkiksi LISP-kielessä; sekä tyypitetetyt lambdakalkyylit.