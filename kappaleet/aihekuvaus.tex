Imperatiivisissa ohjelmoinnissa ohjelma esitetään sarjana komentoja, jotka muokkaavat ohjelman tilaa (\textit{state}). Ohjelman tilalla tarkoitetaan ohjelman muuttujien kokoelmaa. Imperatiivisen ohjelmointiparadigman teoreettisena perustana on laskennan malli, joka tunnetaan Turingin koneena. Turingin kone koostuu nauhasta ja lukupäästä sekä tilasiirtymäfunktiosta. Nauha ja lukupää muodostavat yhdessä Turingin koneen tilan, joka vastaa imperatiivisen ohjelman tilaa. Siirtymäfunktio puolestaan vastaa imperatiivista ohjelmakoodia. Turingin kone esittää laskennan sarjana nauhan ja lukupään tilanmuutoksia, jotka siirtymäfunktio määrittelee. Juuri laskennan esittäminen sarjana tilanmuutoksia karakterisoi parhaiten imperatiivisen ohjelmointiparadigman~\cite[p.~3]{Hudak89}.
\par
Deklaratiivisessa ohjelmoinnissa ohjelman esitys on joukko lausekkeita. Ohjelman suorittaminen merkitsee lausekkeiden arvojen laskemista. Huomattavin ero deklaratiivisen ja imperatiivisen paradigman välillä on se, että deklaratiiviset ohjelmat ovat tilattomia. Eräs tilattomuuden suurista eduista suhteessa imperatiiviseen ohjelmointityyliin on viittausläpinäkyvyys (\textit{referential transparency}). Lausekkeen viittausläpinäkyvyys tarkoittaa sitä, että lausekkeen mikä tahansa alilauseke voidaan korvata toisella lausekkeella, jonka arvo on sama kuin alilausekkeen arvo, muuttamatta ohjelman semantiikkaa~\cite[p.~5]{Hudak89}.
\par
Funktionaalinen ohjelmointiparadigma on osa deklaratiivista ohjelmointiparadigmaa. Funktionaalisessa ohjelmoinnissa evaluoitavat lausekkeet ovat funktioita. Imperatiivisessa ohjelmoinnissa sanaa funktio käytetään usein synonyyminä aliohjelmalle. Näin lavea määritelmä ei ole yhteensopiva funktion matemaattisen määritelmän kanssa, sillä matemaattisella funktiolla ei ole sivuvaikutuksia ja sen arvo riippuu yksikäsitteisesti funktion argumenteista. Funktionaalisen ohjelmoinnin viitekehyksessä matemaattisiin funktiohin rinnastettavissa olevista aliohjelmista käytetään termiä puhdas funktio (\textit{pure function}) erotuksena sivuvaikutuksia sisältäville tai paluuarvoiltaan monikäsitteisille aliohjelmille, joista vastaavasti käytetään termiä ei-puhdas funktio.  
\par
Funktionaalisen ohjelmoinnin teoreettisena perustana on Alonzo Churchin 1930-luvulla kehittämä lambdakalkyyli ~\cite[p.~37--50]{PJ1987}. Lambdakalkyyli on laskennan malli, jossa laskennalla tarkoitetaan annetun lambdalausekkeen arvon laskemista käyttäen lamdalausekkeille määriteltyjä reduktiosääntöjä. Lambdalauseke on joko muuttujasymboli, lambda-abstraktio tai lambdalausekkeen aplikaatio toiselle lambdalausekkeelle ~\cite[p.~9--36]{PJ1987}. Lambda-abstraktiot ovat anonyymejä yhden muuttujan funktiota. Lambda-abstraktio voi ottaa argumenttinaan toisen lambda-abstraktion ja myös lambda-abstraktion arvo voi olla lambda-abstraktio. Lambda-abstraktiot ovat siten yksinkertainen malli korkeamman asteen funktioille (\textit{higher order function}).
\par
Funktionaalisessa ohjelmoinnissa on toki kyse muustakin kuin ohjelman tilattomuuden ja viiteläpinäkyvyyden tavoittelusta. Korkeamman asteen funktiot, hahmontunnistus (\textit{pattern-matching}) ja laiska evaluaatio (\textit{lazy evaluation}) ovat eräitä tyypillisiä modernien funktionaalisten kielten ominaisuuksia~\cite{Hudak89}.  
\par
Ohjelmoinnin kannalta tärkeimpiä sivuvaikutuksia ovat I/O-operaatiot, jotka mahdollistavat informaation vaihtamisen ohjelman ja ulkomaailman välillä.
Koska puhtaat funktiot eivät voi aiheuttaa sivuvaikutuksia, on I/O-abstraktioiden toteuttaminen funktionaalisten kielten suunnittelun kannalta haasteellista. Kielissä kuten Scheme ja ML sivuvaikutuksia sisältävää koodia on mahdollista kirjoittaa, joskin sivuvaikutusten tarpeettoman runsasta käyttöä pidetään huonona ohjelmointityylinä~\cite[p.~23]{Hudak89}. Kielet kuten Haskell sen sijaan sisältävät tyyppijärjestelmiensä tasolla tuen monadiselle ohjelmoinnille, jossa sivuvaikutuksia tuottavat funktiot voidaan eristää puhtaista funktiosta kielen modulaarisuudesta tinkimättä~\cite[p.~4--16]{PJ2000}.       