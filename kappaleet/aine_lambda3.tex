\section{Normaalimuoto ja evaluointi}

Edellä määriteltiin sievennyssäännöt, jotka säilyttävät lambdalausekkeen arvon samana. Jotta lambdalausekkeilla voitaisiin suorittaa laskentaa, täytyy lausekkeen arvo ja lauseke kyetä erottamaan toisistaan. Laskennan määritelmä tulisi muotoilla siten, että (pseudokoodia):
\[ 
(\lambda x. 3 * x + 5) 6 \; \rightarrow \; 3 * 6 + 5 \; \rightarrow \; 18 + 5 \; \rightarrow \; 23 
\]
mutta 
\[ 18 + 5 \; \cancel{\rightarrow} \; (\lambda x. 3 * x + 5) 6 \]
Toisin sanottuna laskennalla tulisi olla tietty suunta ja laskennalla tulisi olla yksikäsitteinen päätepiste aina kun laskenta päättyy. Lambdalausekkeen laskennan päätepistettä kutsutaan lausekkeen normaalimuodoksi ja se määritellään seuraavasti~\cite[s.~9]{Hudak89}: 

\begin{maar}[normaalimuoto]
Lambdalauseke $L$ on normaalimuodossa, jos mihinkään sen alilausekkeista ei voida soveltaa $\beta$- tai $\eta$-reduktiota ja sen sidottujen muuttujien nimet ovat aakkosjärjestyksessä vasemmalta oikealle ja lausekkeen vasemmanpuolisin sidottu muuttuja on aakkosjärjestyksessä seuraavana lausekkeen aakkosjärjestyksessä suurimman vapaan muuttujan jälkeen.  
\end{maar}

Usein sidottujen muuttujien uudelleennimeämistä ei ole tapana kirjoittaa auki, vaan lauseke johdetaan "normaalimuotoon modulo $\alpha$-muunnos", eli muotoon, jossa sen alilausekkeisiin ei voida soveltaa $\beta$- tai $\eta$-reduktioita. Tällöin sidottujen muuttujien uudelleennimeäminen normaalimuodon vaatimalla tavalla on varsin suoraviivaista.
\par
Lambdalausekkeiden evaluointi on siis lausekkeen johtamista normaalimuotoon käyttäen $\beta$- ja $\eta$-reduktioita ja $\alpha$-muunnosta (sidottujen muuttujien uudelleennimeäminen) sen alilausekkeisiin. Erilaisia tapoja tämän tekemiseksi kutsutaan evaluointistrategioiksi. Tyypillisiä evaluointistrategioita ovat sovellusjärjestysevaluointi (\textit{applicative-order evaluation}) ja normaalijärjestysevaluointi~\cite[s.~ 10]{Hudak89} (\textit{normal-order evaluation)}.

\begin{alg}[sovellusjärjestysevaluointi] 
Lambdalausekkeen $L$ sovellusjärjestysevaluointi on evaluointistrategia, jossa redusoidaan aina lausekkeen $L$ (sulkujen mukaan) sisimmäistä vasemmanpuolisinta alilauseketta, joka ei ole normaalimuodossa.
\end{alg}

\begin{alg}[normaalijärjestysevaluointi]
Lambdalausekkeen $L$ normaalijärjestysevaluointi on evaluointistrategia, jossa redusoidaan aina lausekkeen $L$ vasemmanpuolisinta, (sulkujen mukaan) ulointa alilauseketta, joka ei ole normaalimuodossa.
\end{alg}

\begin{esim}
Tarkastellaan, mitä tapahtuu lausekkeelle $(\lambda x . y) ((\lambda x . xx) (\lambda x . xx))$, kun siihen sovelletaan eri evaluointistrategioita. Huomataan aluksi, että:  
\[ (\lambda x . xx) (\lambda x . xx) \rightarrow_{\beta} (\lambda x . xx) (\lambda x . xx) \]
joten normaalijärjestysevaluoinnilla:
\[ (\lambda x . y) ((\lambda x . xx) (\lambda x . xx)) \rightarrow_{\beta} y \]
mutta sovellusjärjestysevaluoinnilla:
\[  (\lambda x . y) ((\lambda x . xx) (\lambda x . xx)) \rightarrow_{\beta}
	(\lambda x . y) ((\lambda x . xx) (\lambda x . xx)) \rightarrow_{\beta}
	(\lambda x . y) ((\lambda x . xx) (\lambda x . xx)) \rightarrow_{\beta} \ldots \]
Sovellusjärjestysevaluointi ei siis tässä tapauksessa koskaan löydä lausekkeen normaalimuotoa. Tämän esimerkin perusteella evaluointistrategian valinnalla on merkitystä sen kannalta, päättyykö laskenta vai ei. 
\end{esim}
\par
Seuraavat lambdalausekkeiden normaalimuotoa koskevat tärkeät tulokset tunnetaan Churchin-Rosserin teoreemina: 
\begin{lau}[Churchin-Rosserin teoreemat]
\begin{enumerate} $ $ %\newline
	\item Kullakin lambdalausekkeella on korkeintaan yksi normaalimuoto.
	\item Jos lausekkeella on normaalimuoto, niin normaalijärjestysevaluointi löytää sen. 
\end{enumerate}
\end{lau}
\par
Churchin-Rosserin teoreemien todistuksia löytyy alan kirjallisuudesta lukuisia, esimerkiksi Henk Barendregtin ja Erik Barendsenin luentomateriaalista \textit{Introduction to Lambda Calculus}~\cite[s.~23--30]{HBEB2000}. Churchin-Rosserin teoreema 1 takaa, että lambdalausekkeiden laskenta on ristiriidatonta: mikään laskenta ei voi johtaa kahteen eri lopputulokseen. Toisaalta teoreema 2 takaa, että mikä tahansa lauseke, joka on mahdollista evaluoida, voidaan evaluoida käyttämällä normaalijärjestysevaluointia.
