

\par
Käytännössä funktioden sivuvaikutuksia ei voida täysin välttää jos halutaan, että ohjelma kykenee kommunikoimaan ympäröivän maailman kanssa. On esimerkiksi mahdotonta kirjoittaa funktio, jonka paluuarvo on käyttäjän antama syöte, siten että funktion paluuarvo olisi yksikäsitteinen. Siksi funktionaalisessa ohjelmoinnissa funktiot on hyödyllistä jakaa puhtaisiin ja ei-puhtaisiin funktiohin. Puhtaat funktiot ovat funktiota, joilla ei ole kielen semantiikan kannalta havaittavia sivuvaikutuksia ja niiden paluuarvo on yksikäsitteinen. Monet modernit funktionaaliset kielet tukevat funktioiden explisiittistä jaottelua puhtaisiin ja ei-puhtaisiin funktiohin. Esimerkiksi Haskell-kielen tyyppijärjestelmässä funktion tyyppi ilmaisee onko funktiolla sivuvaikutuksia vai ei. 
\par

\par

Viittauksellisesti läpinäkyvvällä lausekkeella tarkoitetaan lauseketta, jonka mikä tahansa alilauseke voidaan korvata lausekkeella, jonka arvo on sama kuin korvatun alilausekkeen arvo, muuttamatta ohjelman semantiikkaa. Viittauksellisesta läpinäkyvyydestä seuraa, että puhdas funktio on aina deterministinen, eli sen arvo riippuu yksikäsitteisesti sen argumenteista
 Lambdalausekkeet ovat 