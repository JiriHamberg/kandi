\section{Lambdakalkyylin sovelluksia}

\subsection{Kokonaislukuaritmetiikka}
Tähän mennessä lambdalausekkeita on käsitelty niiden loogisten ja syntaktisten piirteiden kautta liittämättä lausekkeisiin minkäänlaista semantiikkaa. Näytetään seuraavaksi, miten lambdalausekkeilla voidaan määritellä kokonaisluvut ja hieman kokonaislukujen aritmetiikkaa.
\par   
Kokonaisluvut voidaan määritellä monin eri tavoin käyttäen lambdalausekkeita. Näistä tunnetuin lienee Churchin alkuperäinen muotoilu~\cite[s.~20]{Sel2013}, jossa
\[ \textbf{0} \equiv \lambda fx . x \]
\[ \textbf{1} \equiv \lambda fx . fx \]
\[ \textbf{2} \equiv \lambda fx . f (fx) \] 
\[ \textbf{3} \equiv \lambda fx . f(f(fx)) \]
\[ \vdots \]
\[ \textbf{N} \equiv \lambda fx . \underbrace{ f ( f \ldots (f }_{ N-\text{kappaletta}} x)) \ldots ) \]
Kokonaisluku $N$ esitetään siis lausekkeena, jonka ensimmäistä parametria on sovellettu $N$-kertaa itselleen.
\par
Käytettäessä luonnollisille luvuille Churchin koodausta, yhteenlaskufunktio \textbf{ADD} voidaan määritellä seuraavasti~\cite[s.~20]{Sel2013}:
\[ \textbf{ADD} \equiv \lambda nmfx . n f (m f x) \] 
sillä: 
\begin{align*} \textbf{ADD N M} &\equiv (\lambda nmfx . n f (m f x)) \textbf{N M} \\ 
&\equiv  (\lambda nmfx . n f (m f x)) \; (\lambda fx . \underbrace{ f ( f \ldots (f }_{ N-\text{kappaletta}} x)) \ldots )) \; (\lambda fx . \underbrace{ f ( f \ldots (f }_{ M-\text{kappaletta}} x)) \ldots ))\\ 
&\rightarrow_{\beta} (\lambda mfx . ((\lambda fx . \underbrace{ f ( f \ldots (f }_{ N-\text{kappaletta}} x)) \ldots ) f (m f x)) \; \lambda fx . \underbrace{ f ( f \ldots (f }_{ M-\text{kappaletta}} x)) \ldots )\\ 
&\rightarrow_{\beta} \lambda fx . ((\lambda fx . \underbrace{ f ( f \ldots (f }_{ N-\text{kappaletta}} x)) \ldots ) f ( (\lambda fx . \underbrace{ f ( f \ldots (f }_{ M-\text{kappaletta}} x)) \ldots )) f x ) \\
&\rightarrow_{\beta} \lambda fx . ((\lambda x . \underbrace{ f ( f \ldots (f }_{ N-\text{kappaletta}} x)) \ldots ) ( (\lambda fx . \underbrace{ f ( f \ldots (f }_{ M-\text{kappaletta}} x)) \ldots )) f x ) \\
&\rightarrow_{\beta}  \lambda fx . \underbrace{ f ( f \ldots (f }_{ N-\text{kappaletta}} ( (\lambda fx . \underbrace{ f ( f \ldots (f }_{ M-\text{kappaletta}} x)) \ldots ) f x ) \ldots )\\
&\rightarrow_{\beta}  \lambda fx . \underbrace{ f ( f \ldots (f }_{ N-\text{kappaletta}} ( (\lambda x . \underbrace{ f ( f \ldots (f }_{ M-\text{kappaletta}} x)) \ldots )  x ) \ldots) \\
&\rightarrow_{\beta}  \lambda fx . \underbrace{ f ( f \ldots (f }_{ N-\text{kappaletta}} (\underbrace{ f ( f \ldots (f }_{ M-\text{kappaletta}} x) \ldots ) \ldots ) \\
&\equiv \lambda fx . \underbrace{ f ( f \ldots (f }_{ N + M-\text{kappaletta}} x) \ldots ) \\
&\equiv \textbf{N + M}  
\end{align*}

Churchin numeraalien kertolaskufunktio \textbf{MUL} voidaan puolestaan määritellä seuraavalla tavalla~\cite[s.~20]{Sel2013}:

\[ \textbf{MUL} \equiv \lambda n m f . n (mf) \]

Identiteetti $\textbf{MUL N M } \twoheadrightarrow_{\beta}\textbf{N} * \textbf{M}$ voidaan johtaa samaan tapaan kuin funktion \textbf{ADD} tapauksessa suoraviivaisesti, joskin vaivalloisesti, $\beta$-reduktiota käyttämällä. 
\par
Näytetään vielä, miten lauseke $3 \; * \; (1  \; + \; 1)$ voidaan esittää ja evaluoida käyttäen Churchin numeraaleja. Puolalaista notaatiota käyttäen lausekkeelle saadaan seuraava esitys:
\[* \; 3 \; (+ \; 1 \; 1) \] 
joka voidaan esittää lambdalausekkeella:
\[ \textbf{MUL 3 }(\textbf{ADD 1 1}) \]
ja siten:
\begin{align*}
\textbf{MUL 3 }(\textbf{ADD 1 1}) &\equiv (\lambda n m f . n (mf))  \; (\lambda f x . f(f(fx))) \;(\textbf{ADD 1 1}) \\
&\twoheadrightarrow_{\beta} (\lambda n m f . n (mf) ) \; (\lambda f x . f(f(fx))) \; \textbf{2} \\
&\rightarrow_{\beta} (\lambda m f . (\lambda f x . f(f(fx))) \; (m f)) \textbf{2}    \\
&\rightarrow_{\beta} \lambda f . (\lambda f x . f(f(fx))) (\textbf{2} \; f) \\
&\equiv \lambda f x . (\lambda f . f(f(fx))) ((\lambda f x . f (f x)) \; f) \\
&\rightarrow_{\beta} \lambda f . (\lambda f x .  f(f(fx))) (\lambda x . f (f x)) \\
&\rightarrow_{\beta} \lambda f . (\lambda x .  (\lambda x . f (f x))((\lambda x . f (f x))((\lambda x . f (f x))x))\\
&\rightarrow_{\beta} \lambda f . (\lambda x .  (\lambda x . f (f x))((f (f( f( f x))))  \\
&\rightarrow_{\beta} \lambda f . (\lambda x . f( f (f (f( f( f x)))))) \\
&\equiv \lambda f x. f( f (f (f( f( f x))))) \\
&\equiv \textbf{6}
\end{align*} 

\subsection{Kontrollinohjaus ja rekursiiviset funktiot}
Tarkastellaan seuraavaksi, miten kontrollinohjaus voidaan toteuttaa lambdakalkyylin keinoin. Määritellään aluksi totuusarvot $true$ ja $false$ lambdalausekkeina:
\[ \textbf{ TRUE } \equiv \lambda x y . x \]
\[ \textbf{ FALSE } \equiv \lambda x y . y \]
Totuusarvo $\textbf{TRUE}$ on siis funktio, joka ottaa kaksi argumenttia ja palauttaa näistä ensimmäisen. Totuusarvo $\textbf{FALSE}$ on funktio, joka ottaa kaksi argumenttia ja palauttaa jälkimmäisen argumenttinsa. On syytä huomata, että tässä käytetty totuusarvojen määritelmä on täysin mielivaltainen. Yllä annettujen määritelmien etuna on se, että ehtolauseke \textbf{IF} voidaan antaa hyvin kompaktissa muodossa:
\[ \textbf{IF} \equiv \lambda p x y . p x y \]  
Lauseke $\textbf{IF}$ on funktio, joka ottaa argumentteinaan totuusarvon $p$ ja kaksi mielivaltaista lambdalauseketta $x$ ja $y$, ja palauttaa totuusarvon $p$, johon on sovellettu lausekkeita $x$ ja $y$. Suoraan yllä olevista totuusarvojen määritelmistä seuraa, että jos totuusarvo $p$ on $\textbf{TRUE}$, niin palautetaan lauseke $x$, ja jos $p$ on $\textbf{FALSE}$, niin palautetaan lauseke $y$.
\par 
Määritellään vielä yksi ehtolausekkeisiin liittyvä funktio, nimittäin vertailuoperaattori $\textbf{ISZERO}$, joka ottaa argumenttinaan Churchin numeraalin ja palauttaa lausekkeen $\textbf{TRUE}$, jos lauseke on $\textbf{0}$, ja lausekkeen $\textbf{FALSE}$ muulloin. $\textbf{ISZERO}$ voidaan määritellä seuraavasti:
\[\textbf{ ISZERO } \equiv \lambda n . n (\lambda m . \textbf{ FALSE }) \textbf{ TRUE }  \]
sillä:
\begin{align*}
\textbf{ ISZERO } \textbf{ N }
&\equiv (\lambda n . n (\lambda m . \textbf{ FALSE }) \textbf{ TRUE }) \textbf{ N } \\
&\rightarrow_{\beta} \textbf{ N } (\lambda m . \textbf{ FALSE }) \textbf{ TRUE } \\
&\equiv (\lambda fx . \underbrace{ f ( f \ldots (f }_{ N-\text{kappaletta}} x)) \ldots )  (\lambda m . \textbf{ FALSE }) \textbf{ TRUE } \\
&\rightarrow_{\beta} (\lambda x . \underbrace{ (\lambda m . \textbf{ FALSE }) ( (\lambda m . \textbf{ FALSE }) \ldots ((\lambda m . \textbf{ FALSE }) }_{ N-\text{kappaletta}} x)) \ldots ) \textbf{ TRUE }
\end{align*} 
Nyt jos $\textbf{N} = \textbf{0}$, niin lauseke pelkistyy muotoon:
\[ (\lambda x . x) \textbf{ TRUE } \rightarrow_{\beta} \textbf{TRUE} \]
Ja jos $\textbf{N} > \textbf{0}$, niin:
\begin{align*}
\textbf{ ISZERO } \textbf{ N } &\twoheadrightarrow_{\beta} (\lambda x . \underbrace{ (\lambda m . \textbf{ FALSE }) ( (\lambda m . \textbf{ FALSE }) \ldots ((\lambda m . \textbf{ FALSE }) }_{ N-\text{kappaletta}} x)) \ldots ) \textbf{ TRUE } \\
&\rightarrow_{\beta}  (\lambda m . \textbf{ FALSE }) ( (\lambda m . \textbf{ FALSE }) \ldots ((\lambda m . \textbf{ FALSE }) \textbf{ TRUE }) )) \ldots ) \\
&\vdots \\
&\rightarrow_{\beta} \textbf{ FALSE }
\end{align*}
\par
Kuten yllä nähtiin, erinäisten ehtolausekkeiden määritteleminen lambdalausekkeiden avulla on varsin suoraviivaista. Huomattavasti haastavampaa on toistolausekkeiden toteuttaminen. Koska lambdalausekkeita voidaan käsitellä monella tapaa kuten funktioita, olisi toisto luontevaa toteuttaa rekursion avulla. Keskeiseksi ongelmaksi osoittautuu, että lambdakalkyylissa lausekkeita ei voida sitoa nimiin. Näinollen lambda-abstraktion rungossa ei voida viitata lambda-abstraktioon itseensä, mikä mahdollistaisi rekursion toteuttamisen triviaalisti. 
\par 
Rekursion ongelma on ratkaistavissa niin kutsuttujen kiintopistelausekkeiden (\textit{fixed-point combinator}) avulla. Tarkastellaan lauseketta: 
\[ \textbf{ Y } \equiv \lambda f . (\lambda x . f(xx)) (\lambda x . f(xx)) \]
Tämä lauseke tunnetaan Curryn kiintopistelausekkeena löytäjänsä Haskell Curryn mukaan. Curryn kiintopistelausekkeella on seuraava mielenkiintoinen ominaisuus:
\begin{align*}
\textbf{ Y } e &\equiv \lambda f . (\lambda x . f(xx)) (\lambda x . f(xx)) e \\
&\rightarrow_{\beta} (\lambda x . e(xx)) (\lambda x . e(xx)) \\
&\rightarrow_{\beta} e((\lambda x . e(xx)(\lambda x . e(xx)) \\
&\ =_{\beta} e(\textbf{Y }e) 
\end{align*}
Kun siis kiintopistelausekkeeseen sovelletaan lambdalauseketta $e$, niin saadaan lambdalauseke $e$, johon on sovellettu kiintopistelauseketta, johon on sovellettu lauseketta $e$. Kiintopistelausekkeen avulla voidaan tästä syystä simuloida rekursiota, kuten pian nähdään.
\par
Kertomafunktiolle voidaan antaa seuraava rekursiivinen määritelmä:

\begin{align*}
factorial(0) &= 1 \\ 
factorial(n) &= n * factorial(n-1) \text{ kun } n > 0
\end{align*}

Yritetään muodostaa lambdalauseke, jolla voidaan laskea luonnollisen luvun $n$ kertoma edellisen määritelmän hengessä. Oletetaan tunnetuksi lambdalauseke \textbf{PRED}~\cite[s.~8]{Smolka2013}, joka ottaa argumenttinaan Churchin numeraalin \textbf{N} ja palauttaa Churchin numeraalin \textbf{N - 1}, jos \textbf{N} on vähintään 1, muulloin lausekkeen \textbf{0}. Määritellään:
\[ \textbf{FACT} \equiv \lambda f n. \textbf{IF } (\textbf{ISZERO } n) \textbf{ 1 } (\textbf{MUL } n (f (\textbf{PRED } n))) \]
Nyt luvun $n$ kertoma voidaan laskea soveltamalla yllä annettua lauseketta \textbf{FACT} Curryn kiintopistelausekkeeseen. Seuraavassa esimerkissä on hahmotelma luvun 2 kertoman laskemiseksi käyttäen Curryn kiintopistelauseketta ja yllä annettua lauseketta \textbf{FACT}. Kuten huomataan, on yksinkertaisenkin rekursiivisen lausekkeen saattaminen normaalimuotoon varsin työlästä.  

\begin{esim}[Kertomafunktion evaluointi]
\begin{align*}
\textbf{Y FACT 2} &\equiv \textbf{Y }\lambda f n. \textbf{IF } (\textbf{ISZERO } n) \textbf{ 1 } (\textbf{MUL } n  (f (\textbf{PRED } n))) \textbf{ 2} \\
&\twoheadrightarrow_{\beta} \textbf{FACT }( \textbf{Y } (\lambda f n. \textbf{IF } (\textbf{ISZERO } n ) \textbf{ 1 } (\textbf{MUL } n (f (\textbf{PRED } n))) ) \textbf{ 2} \\
&\equiv \lambda f n. \textbf{IF } (\textbf{ISZERO } n) \textbf{ 1 } (\textbf{MUL } n  (f (\textbf{PRED } n))) ) \\
	& \hspace{10 mm} ( \textbf{Y } (\lambda f n. \textbf{IF } (\textbf{ISZERO } n ) \textbf{ 1 } (\textbf{MUL } n (f (\textbf{PRED } 	n))) ) \textbf{ 2 }\\
&\rightarrow_{\beta} (\lambda n. \textbf{IF } (\textbf{ISZERO } n) \textbf{ 1 } (\textbf{MUL } n \\
	& \hspace{10 mm} (( \textbf{Y }\lambda f n'. \textbf{IF } (\textbf{ISZERO } n' ) \textbf{ 1 } (\textbf{MUL } n' (f (\textbf{PRED  	} n'))) )  (\textbf{PRED } n))) ) \textbf{ 2 }\\
&\rightarrow_{\beta} \textbf{IF } (\textbf{ISZERO } 2) \textbf{ 1 } (\textbf{MUL 2 } \\
	& \hspace{10 mm} (( \textbf{Y }\lambda f n'. \textbf{IF } (\textbf{ISZERO } n' ) \textbf{ 1 } (\textbf{MUL } n'(f (\textbf{PRED  	} n'))) )  (\textbf{PRED 2 } ))) ) \\
&\twoheadrightarrow_{\beta} \textbf{MUL 2 } \\
	& \hspace{10 mm} (( \textbf{Y }\lambda f n'. \textbf{IF } (\textbf{ISZERO } n' ) \textbf{ 1 } (\textbf{MUL } n' (f (\textbf{PRED  	} n'))) )  (\textbf{PRED 2 }))) )	\\
&\twoheadrightarrow_{\beta} \textbf{MUL 2 } \\
	& \hspace{10 mm} (\textbf{FACT } (( \textbf{Y }\lambda f n'. \textbf{IF } (\textbf{ISZERO } n' ) \textbf{ 1 } (\textbf{MUL } n' (f (\textbf{PRED } n'))) )  (\textbf{1 } ))) ) \\
& \equiv \textbf{MUL 2 } (\textbf{ FACT } ( \textbf{Y FACT} )  \textbf{1}) \\
&\vdots \\
& \twoheadrightarrow_{\beta} \textbf{MUL 2 } (\textbf{MUL 1 } ( \textbf{FACT} (\textbf{Y FACT}) \textbf{0} ) ) \\
&\vdots \\
& \twoheadrightarrow_{\beta} \textbf{MUL 2 } (\textbf{MUL 1 } \textbf{1} )\\
&\vdots \\
& \twoheadrightarrow_{\beta} \textbf{ 2 }
\end{align*}
\end{esim}
\par


