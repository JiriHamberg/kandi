\section{Lambdakalkyylin sovelluksia}

\subsection{Kokonaisluku aritmetiikka}
Tähän mennessä olemme käsitelleet lambdalausekkeiden loogisia ja syntaktisia piirteitä liittämättä lausekkeisiin minkäänlaista semantiikkaa. Näytetään seuraavaksi miten lambdalausekkeilla voidaan määritellä kokonaisluvut ja hieman kokonaislukujen aritmetiikkaa.
   
\begin{esim}[Churchin numeraalit ja aritmetiikka]
Kokonaisluvut voidaan määritellä monin eri tavoin käyttäen lambdalausekkeita. Näistä tunnetuin lienee Churchin alkuperäinen muotoilu~\cite[s.~20]{Sel2013}, jossa

\[ \textbf{0} \equiv \lambda fx . x \]
\[ \textbf{1} \equiv \lambda fx . fx \]
\[ \textbf{2} \equiv \lambda fx . f (fx) \] 
\[ \textbf{3} \equiv \lambda fx . f(f(fx)) \]
\[ \vdots \]
\[ \textbf{N} \equiv \lambda fx . \underbrace{ f ( f \ldots (f }_{ N-\text{kappaletta}} x)) \ldots ) \]
 
Kokonaisluku $N$ esitetään siis lausekkeena, jonka ensimmäistä parametria on sovellettu $N$-kertaa itselleen.

\par

Nyt yhteenlasku-funktio  \textbf{ADD} voidaan määritellä seuraavasti~\cite[s.~20]{Sel2013}:

\[ \textbf{ADD} \equiv \lambda nmfx . n f (m f x) \] 

sillä 

%\begin{align}
%\textbf{PLUS} N M &\equiv (\lambda fx . \underbrace{ f ( f \ldots (f }_{ N-\text{kappaletta}}) \\
%&= foo bar
%\end{align}

\begin{align*} \textbf{ADD N M} &\equiv (\lambda nmfx . n f (m f x)) \textbf{N M} \\ 
&\equiv  (\lambda nmfx . n f (m f x)) \; (\lambda fx . \underbrace{ f ( f \ldots (f }_{ N-\text{kappaletta}} x)) \ldots )) \; (\lambda fx . \underbrace{ f ( f \ldots (f }_{ M-\text{kappaletta}} x)) \ldots ))\\ 
&\rightarrow_{\beta} (\lambda mfx . ((\lambda fx . \underbrace{ f ( f \ldots (f }_{ N-\text{kappaletta}} x)) \ldots ) f (m f x)) \; \lambda fx . \underbrace{ f ( f \ldots (f }_{ M-\text{kappaletta}} x)) \ldots )\\ 
&\rightarrow_{\beta} \lambda fx . ((\lambda fx . \underbrace{ f ( f \ldots (f }_{ N-\text{kappaletta}} x)) \ldots ) f ( (\lambda fx . \underbrace{ f ( f \ldots (f }_{ M-\text{kappaletta}} x)) \ldots )) f x ) \\
&\rightarrow_{\beta} \lambda fx . ((\lambda x . \underbrace{ f ( f \ldots (f }_{ N-\text{kappaletta}} x)) \ldots ) ( (\lambda fx . \underbrace{ f ( f \ldots (f }_{ M-\text{kappaletta}} x)) \ldots )) f x ) \\
&\rightarrow_{\beta}  \lambda fx . \underbrace{ f ( f \ldots (f }_{ N-\text{kappaletta}} ( (\lambda fx . \underbrace{ f ( f \ldots (f }_{ M-\text{kappaletta}} x)) \ldots ) f x ) \ldots )\\
&\rightarrow_{\beta}  \lambda fx . \underbrace{ f ( f \ldots (f }_{ N-\text{kappaletta}} ( (\lambda x . \underbrace{ f ( f \ldots (f }_{ M-\text{kappaletta}} x)) \ldots )  x ) \ldots) \\
&\rightarrow_{\beta}  \lambda fx . \underbrace{ f ( f \ldots (f }_{ N-\text{kappaletta}} (\underbrace{ f ( f \ldots (f }_{ M-\text{kappaletta}} x) \ldots ) \ldots ) \\
&\equiv \lambda fx . \underbrace{ f ( f \ldots (f }_{ N + M-\text{kappaletta}} x) \ldots ) \\
&\equiv \textbf{N + M}  
\end{align*}

Kertolasku-funktiolle \textbf{MUL} voidaan antaa seuraava määritelmä~\cite[s.~20]{Sel2013}:

\[ \textbf{MUL} \equiv \lambda n m f . n (mf) \]

Identiteetti $\textbf{MUL N M } \equiv \textbf{N} * \textbf{M}$ voidaan johtaa samaan tapaan kuin funktion \textbf{ADD} tapauksessa suoraviivaisesti (joskin vaivalloisesti) $\beta$-reduktiota käyttämällä. 

\par

Näytetään vielä miten lauseke $3 * (1+1)$ voidaan esittää ja evaluida käyttäen Churchin numeraaleja. Ensinnäkin käyttäen Puolalaista notaatiota, lausekkeelle saadaan seuraava muotoilu
\[* \; 3 \; (+ \; 1 \; 1) \] 
joka voidaan suoraviivaisesti muuntaa lambdalausekkeeksi
\[ \textbf{MUL 3 }(\textbf{ADD 1 1}) \]
ja siten
\begin{align*}
\textbf{MUL 3 }(\textbf{ADD 1 1}) &\equiv (\lambda n m f . n (mf))  \; (\lambda f x . f(f(fx))) \;(\textbf{ADD 1 1}) \\
&\twoheadrightarrow_{\beta} (\lambda n m f . n (mf) ) \; (\lambda f x . f(f(fx))) \; \textbf{2} \\
&\rightarrow_{\beta} (\lambda m f . (\lambda f x . f(f(fx))) \; (m f)) \textbf{2}    \\
&\rightarrow_{\beta} \lambda f . (\lambda f x . f(f(fx))) (\textbf{2} \; f) \\
&\equiv \lambda f x . (\lambda f . f(f(fx))) ((\lambda f x . f (f x)) \; f) \\
&\rightarrow_{\beta} \lambda f . (\lambda f x .  f(f(fx))) (\lambda x . f (f x)) \\
&\rightarrow_{\beta} \lambda f . (\lambda x .  (\lambda x . f (f x))((\lambda x . f (f x))((\lambda x . f (f x))x))\\
&\rightarrow_{\beta} \lambda f . (\lambda x .  (\lambda x . f (f x))((f (f( f( f x))))  \\
&\rightarrow_{\beta} \lambda f . (\lambda x . f( f (f (f( f( f x)))))) \\
&\equiv \lambda f x. f( f (f (f( f( f x))))) \\
&\equiv \textbf{6}
\end{align*} 
\end{esim}   
   