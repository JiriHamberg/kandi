\section{Tekstinäyte}

Tämän työn tarkoituksena tutkia funktionaalisen ohjelmoinnin teoreettisia perusteita ja selvittää, miten funktionaalisia ohjelmointikieliä toteutetaan. Luvussa 1 määrittelemme mitä funktionaalinen ohjelmointiparadigma tämän työn viitekehyksessä tarkoittaaa. Käymme myös lyhyesti läpi funktionaalisen ohjelmoinnin historiaa ja esittelemme lyhyesti eräitä merkittäviä funktionaalisia ohjelmointikieliä ja funktionaalisiin kieliin liittyviä käsitteitä. 

Luvussa 2 etenemme käsittelemään tyypitöntä lambdakalkyylia. Kerromme lyhyesti lambdakalkyylin historiasta ja merkityksestä funktionaalisen paradigman synnylle. Näytämme erityisesti miten lambdakalkyyli voidaan tulkita paitsi matemaattisena viitekehyksenä ja laskennan mallina, myös yksinkertaisena funktionaalisena ohjelmointikielenä. 

Luvussa 3 tarkastelemme LISP-ohjelmointikieltä. Käsittelemme LISP-kielen matemaattista formalismia mukaanlukien S-lausekkeet, funktio-muodot ja anonyymit funktiot. Näytämme lopuksi miten yksinkertaisen LISP-tulkin voisi toteuttaa apply- ja eval-funktiota hyödyntäen. Pyrimme löytämään yhteneväisyyksiä lambdakalkyylin ja LISP-kielen formalismin väliltä.

Luvussa 4 perehdymme modernien funktionaalisten kielten, kuten ML ja Haskell, erityispiirteisiin. Käsittelemme laiskaa evaluaatiota, algebrallisia tietotyyppejä, vahvan ja staattista tyypityksen toteuttamista funktionaalissa ohjelmointikielissä sekä tyyppipäättelyä. Erittelemme haasteita joita edellämainitut ominaisuudet asettavat ohjelmointikielen toteuttamiselle ja esitämme ratkaisuja näihin ongelmiin.