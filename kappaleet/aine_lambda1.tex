\section{Lambdalausekkeet, vapaat muuttujat ja reduktiosäännöt}

Lambdalausekkeita ovat muodollisesti muuttujasymbolit, lambda-abstraktiot ja lambdalausekkeen sovellus lambdalausekkeelle. Formaalisti tämä määritelmä voidaan muotoilla seuraavasti~\cite[s.~8]{Hudak89}:

\begin{maar}[lambdalausekkeet]
Lambdalausekkeiden joukko $E$ määritellään rekursiivisesti: 
\[ V \subset E \]
\[ \text{Jos } e_{1} \in E \text{ ja } e_{2} \in E, \text{ niin }  (e_{1} \; e_{2}) \in E \]
\[ \text{Jos } x \in V \text{ ja } e \in E, \text{ niin } \lambda x.e \in E \]

missä $V$ on ääretön joukko muuttujia. Muuttujia on tapana merkitä symboleilla $x, y,z...$ tai vaihtoehtoisesti $x_{1}, x_{2}, x_{3}...$ ja mielivaltaisia lambdalausekkeita puolestaan symboleilla $L, N, M ...$ tai vaihtoehtoisesti $e_{1}, e_{2}, e_{3}...$.  Lauseketta joka on muotoa $\lambda x.e$ kutsutaan lambda-abstraktioksi tai lyhyesti abstraktioksi. Lauseketta, joka on muotoa $(e_{1} \; e_{2})$ kutsutaan lausekkeen $e_{2}$ sovellukseksi $e_{1}$:lle tai lyhyesti sovellukseksi.
\end{maar}

Sovelluksille on tapana käyttää vasemmalle assosioivaa lyhennysmerkintää: 
\[e_{1} e_{2} e_{3} \equiv ((e_{1} \; e_{2}) \; e_{3})\]
Abstraktioille puolestaan käytetään seuraavaa oikealle assosiovaa lyhennysmerkintää: 
\[ \lambda x_{1}x_{2}...x_{n}.e \equiv (\lambda x_{1} . ( \lambda x_{2} . ( \: ... \: ( \lambda x_{n} . e ))) \: ... \: ) \]

\par
Esimerkiksi seuraavat ovat lambdalausekkeita:
\pagebreak
\[ x \]
\[ xyz \]
\[ \lambda x . xyz \]
\[ (\lambda xy . xyx) z \]

%Lambdalausekkeita reduktiosääntöjen kannalta tärkeitä käsitteitä ovat lambdalausekkeiden vapaat ja sidotut muuttujat sekä %substituutiosäännöt. Lambdalausekkeiden vapaat muuttujat ovat intuitiivisesti hyvin samanlaisia kuin ohjelmointikielissä yleensäkin: %muuttujien sidonnat vaikuttavat lausekehierarkiassa ylhäältä alaspäin ja hierarkiassa alempana sijaitsevat sidonnat sitovat vahvemmin kuin %hierarkiassa ylempänä sijaitsevat sidonnat.   

Lambdalausekkeen vapailla muuttujilla tarkoitetaan sellaisia lausekkeessa esiintyviä muuttujia, jotka eivät esiinny lausekkeessa lambda-abstraktion muuttujaosana~\cite[s.~8]{Hudak89}. Formaalisti: 

\begin{maar}[Vapaat ja sidutut muuttujat]
Lambdalausekkeen $e$ vapaat muuttujat, joita merkitään $fv(e)$, määritellään seuraavasti: 
\[fv(x) = \{x\}\ \text{ jos } x \text{ on muuttuja} \]
\[fv(e_{1}e_{2}) = fv(e_{1}) \cup fv(e_{2}) \]
\[ fv(\lambda x.e) = fv(e) - \{x\} \]

Siis muuttuja $x$ on vapaa lambdalausekkeessa $e$ jos $x \in fv(e)$, muuten $x$ on sidottu lambdalausekkeessa $e$.
\end{maar} 
\par
Jotta lambdalausekkeiden reduktiosäännöt voidaan määritellä, tarvitaan avuksi vielä seuraavat substituutiosäännöt ~\cite[s.~8]{Hudak89}.

\begin{maar}[substituutiosäännöt]
Olkoon $e$ lambdalauseke ja $x$ muuttuja. Muuttujan $x$ substituutiota lambdalausekkeella $e$ lambdalausekkeessa $e_{1}$ merkitään $[e/x] e_{1}$ ja se määritellään rekursiivisesti:  

\[[e/x]x_{1} = 
        \begin{cases}
                e & \text{jos } x = x_{1} \\
                x_{1} & \text{muulloin}
        \end{cases}
        \label{eq:S1} \tag{\textbf{S1}}
\]
\[ [e/x](e_{1} \; e_{2}) = ([e/x]e_{1} \; [e/x]e_{2}) \label{eq:S2} \tag{\textbf{S2}}\]
\[[e/x](\lambda x_{1}.e_{1}) = 
        \begin{cases}
                \lambda x_{1}.e_{1} & \text{jos } x = x_{1} \\
                \lambda x_{1}.[e/x]e_{1} & \text{ jos } x \neq x_{1} \text{ ja } x_{1} \notin fv(e) \\
                \lambda x_{i}.[e/x]([x_{i}/x_{1}]e_{1}) & \text{muulloin, missä } x_{i} \neq x \text { ja } x_{i} \neq x_{1} \text{ ja } x_{i} \notin fv(e) \cup fv(e_{1})
        \end{cases}
 \label{eq:S3} \tag{\textbf{S3}}       
\]
\end{maar} 

Määritelmän olennaisin sisältö on se, että substituution kohteena olevan lambdalausekkeen sidotut muuttujat tulee uudelleennimetä enen sijoituksen tekemistä siinä tapauksessa että substituution suorittaminen muuttaisi substituution kohteena olevan lausekkeen tai sijoitettavan lausekkeen vapaita muuttujia sidotuiksi. Esimerkiksi: \\

\[ [\lambda x . x / y]((\lambda x . yx)  \stackrel{\eqref{eq:S3}}{=} \\ 
	\lambda z . [\lambda x . x / y](yz) \stackrel{\eqref{eq:S2}}{=} \\
	\lambda z . [\lambda x . x / y]y [\lambda x . x / y]z \stackrel{\eqref{eq:S1}}{=} \lambda z . (\lambda x . x) z	
\]   

\par

Määritellään lambdalausekkeiden reduktiosäännöt ~\cite[s.~9]{Hudak89}. Nämä säännöt kertovat miten lambdalausekkeita on sallittua sieventää. Beta-reduktio ilmaisee miten lambda-abstraktiosta ja mielivaltaisesta lambdalausekkeesta koostuvaa sovellusta voidaan sieventää muuttamatta lausekkeen arvoa. Eta-reduktiota puolestaan ilmaisee sen, että jos abstraktio ei tee muuta kuin soveltaa muuttujaansa lambdalausekkeeseen $e$, niin abstraktion arvo on sama kuin lausekkeen $e$.     

\begin{maar}[reduktiosäännöt]	
	%\begin{enumerate}		
\[\beta \text{-reduktio: } \; (\lambda x.e )\; e_{1} \rightarrow_{\beta} [e_{1} / x]e \]
\[\eta \text{-reduktios: } \; \lambda x.(e \; x) \rightarrow_{\eta} e \text{ jos } x \notin fv(e) \]	
	%\end{enumerate}
Jos $\beta$-reduktion toistuva soveltaminen lausekkeen $N$ alilausekkeisiin tuottaa lausekkeen $M$, niin sanotaan, että $M$ on lausekkeen $N$ \textbf{beta-reduktio} ja merkitään $N  \twoheadrightarrow_{\beta} M$. Vastaavasti Jos $\eta$-reduktion toistuva soveltaminen lausekkeen $N$ alilausekkeisiin tuottaa lausekkeen $M$, niin sanotaan, että $M$ on lausekkeen $N$ \textbf{eta-reduktio} ja merkitään $N  \twoheadrightarrow_{\eta} M$. Lisäksi, jos lauseke $M$ saadaan lausekkeesta $N$ soveltamalla sen alilausekkeisiin toistuvasti eta- tai beta-reduktiota, niin sanotaan, että $M$ on lausekkeen $N$ \textbf{reduktio} ja merkitään $N  \twoheadrightarrow M$        
\end{maar}

\par

%\begin{esim}[$\beta$-reduktio]
%<ESIMERKKI>
%\end{esim}
%
%\par

Seuraavat relaatiot~\cite[s.~23--24]{HBEB2000} karakterisoivat lambdalausekkeiden arvojen yhtäsuuruuden ja niiden ideana on toimia yksinkertaisina sääntöinä, joiden avulla voidaan päätellä ovatko kaksi lambdalausekkeen arvoiltaan yhtäsuuret. Kaksi lambdalauseketta ovat alfa-ekvivalentit, mikäli ne ne eroavat toisistaan ainoastaan sidottujen muuttijiensa nimien suhteen. Kaksi lauseketta ovat beta- tai eta-ekvivalentit jos toinen niistä saadaan toisesta käyttämällä beta- tai eta-reduktiota. Lisäksi kaksi lauseketta on keskenään alfa-ekvivalentit, mikäli toinen saadaan toisesta vaihtamalla sidottujen muuttujien nimiä.

\begin{maar}[lambdalausekkeiden arvojen yhtäsuuruus]
	%\begin{enumerate}
		 
\[ 
	\alpha \text{-ekvivalenssi: } \; \lambda x.e =_{\alpha} \lambda x_{i}.		[x/x_{i}]e \text{,  } \forall x_{i} \notin fv(e) 
\]
\[ 
	\beta \text{-ekvivalenssi: } \; N =_{\beta} M \text{ jos ja vain jos } N 		\twoheadrightarrow_{\beta} M \text{ tai }
	M \twoheadrightarrow_{\beta} N 
\]
\[ 
	\eta \text{-ekvivalenssi: } \; N =_{\eta} M \text{ jos ja vain jos } N 		\twoheadrightarrow_{\eta} M \text{ tai }
	M \twoheadrightarrow_{\eta} N 
\]
	%\end{enumerate}
Lambdalausekkeet $N$ ja $M$ ovat arvoltaan yhtäsuuret, mikäli	on olemassa lambdalauseke $L$ siten että $N$ ja $L$ ovat keskenään alfa- beta- tai eta-ekvivalenttit ja $M$ ja $L$ ovat keskenään alfa-, beta- tai eta-ekvivalentit.	
\end{maar}

Intuitiivisesti edellinen määritelmä tarkoittaa, että kaksi lambdalauseketta ovat arvoiltaan yhtäsuuret, mikäli on olemassa lauseke, joka voidaan johtaa molemmista lausekkeista käyttäen eta- ja beta-reduktiota sekä uudelleennimeämällä sidottuja muuttujia. 

%\par

 %Esimerkiksi seuraavat lambdalausekkeet ovat arvoiltaan yhtäsuuret:
 
%Lambdalausekkeen sanotaan olevan normaalimuodossa, jos siihen ei voida soveltaa $\eta$- tai $\beta$-reduktioita. Voidaan osoittaa, että kaikille lambdalausekkeille, jotka voidaan redusoida normaalimuotoon, pätee, että normaalimuoto saavutetaan valitsemalla aina lausekkeen vasemmanpuolisin alilauseke, jota voidaan vielä redusoida ja soveltamalla siihen $\eta$- tai $\beta$-reduktiota ja toistamalla tätä menettylyä kunnes lauseke on normaalimuodossa. Lisäksi voidaan osoittaa että jos lauseke voidaan redusoida normaalimuotoon, niin lausekkeen normaalimuoto on yksikäsitteinen. Näinollen lambdalausekkeen arvon laskemisella tarkoitetaan lausekkeen redusoimista normaalimuotoon.
%\par
%Kaikille lambdalausekkeille ei ole olemassa normaalimuotoa. Alan Turing todisti vuonna 1937, että Turingin kone on laskennan mallina ekvivalentti lambdakalkyylin kanssa. Tämä tulos ei voisi pitää paikkaansa, jos jokaiselle lambdalausekkeelle olisi löydettävissä normaalimuoto. Muuten ratkeamattomaksi tunnettu pysähtymisongelma voitaisiin ratkaista simuloimalla Turingin koneita lambdalausekkeilla.
%\par 
%Vaikka lambdalausekkeet ovat rakenteeltaan melko yksinkertaisia, voidaan niiden avulla ymmärtää ja mallintaa monia funktionaalisen ohjelmoinnin kannalta tärkeitä käsitteitä. Koska lambda-abstraktiot voivat ottaa parametrinaan ja antaa paluuarvonaan lambdafunktion, ovat esimerkiksi korkeamman asteen funktiot varsin luonnollinen osa lambdakalkyylia.
