\documentclass[12pt]{beamer}

\usepackage[finnish]{babel}
\usepackage[utf8]{inputenc}
\usepackage{amsfonts,amsmath,amssymb,amsthm,booktabs,color,enumitem,graphicx}

\begin{document}


\title{Lambdakalkyyli}
\author{Jiri Hamberg}
\titlegraphic{\includegraphics[scale=0.4]{xkcd-functional.png}}
\frame{\maketitle}

\begin{frame}
\frametitle{Historiaa}

\begin{itemize}
\item loogikko David Hilbert esitti vuonna 1928 nimellä Entscheidungsproblem tunnetun haasteen:
	\begin{itemize}
	\item anna algoritmi, joka ottaa syötteenään joukon aksioomia ja lauseen ja kertoo voidaanko lause päätellä annetuista aksioomista
	\item \alert{ongelma: miten määritellä algoritmi ja laskettavuus?}
\pause	
	\end{itemize}
\item vuonna 1936 Church ja Turing esittivät omat ratkaisuehdotuksensa Hilbertin haasteeseen
	\begin{itemize}
	\item Alonzo Church: lambdakalkyyli	
	\item Alan Turing: Turingin kone
	\end{itemize}
\pause
\item Turing myös todisti että lambdakalkyyli $\Leftrightarrow$ Turingin koneet  
	\begin{itemize} \item \alert{Churchin-Turingin teesi} \end{itemize} 
\end{itemize}
\end{frame}

\begin{frame}
\frametitle{Lambdalausekkeet}
Tavoite:
\[ 
(\lambda x \;. \; 3x + 5) \; 6 \; \rightarrow \; 3 * 6 + 5 \; \rightarrow \; 18 + 5 \; \rightarrow \; 23 
\]
\pause
\[ 
(\lambda x \;. \; 2x) \; (\lambda x \;. \; x + 3) \; \rightarrow \; \lambda x \; . \; 2(x + 3) 
\]
\end{frame}

\begin{frame}
\frametitle{Lambdalausekkeet}
Määritelmä: \\
Lambdalausekkeiden joukko $E$:

\[ V \subset E \textbf{  (muuttuja)}\]
\[ \text{Jos } e_{1} \in E \text{ ja } e_{2} \in E, \text{ niin }  (e_{1}e_{2}) \in E \textbf{  (sovellus)} \]
\[ \text{Jos } x \in V \text{ ja } e \in E, \text{ niin } \lambda x.e \in E \textbf{  (abstraktio)}\]

missä $V$ on numeroituva symbolijoukko, esimerkiksi ASCII-merkkijonojen joukko.

\pause
\par
Esimerkkejä:
\[ ((xy)z)  \Leftrightarrow xyz \textbf{  (Notaatio)} \]
\[ \lambda x . xyz \]
\[ (\lambda x . (\lambda y . xyx)) z \Leftrightarrow \lambda xy . xyx \textbf{  (Notaatio, Curryn muunnon) } \]
\par
\pause
\alert{Määritelmä $\rightarrow$ Tavoite ???}
\end{frame}

\begin{frame}
\frametitle{Reduktiot ja $\alpha$-konversio}
\begin{itemize}
\item lambdalausekkeita voidaan "sieventää" käyttäen $\beta$- ja $\eta$-reduktiota sekä $\alpha$-konversiota
\pause
\item $\alpha$-konversio uudelleennimeää lausekkeen sidottuja muuttujia:
\[ \lambda x . xyz  \rightarrow_{\alpha} \lambda w . wyz  \]
\pause
\item $\beta$-reduktio "sijoittaa" sovelletun lausekkeen lambda-abstraktion muuttujaan:
\[ (\lambda xy . xyx) z  \rightarrow_{\beta} \lambda y . zyz \]
\pause
\item $\eta$-reduktio poistaa "turhan" lambda-abstraktion:
\[ \lambda xy . (yzy) x \rightarrow_{\eta} \lambda y . yzy \]
\end{itemize}
\end{frame}

\begin{frame}
\frametitle{Churchin numeraalit}
\begin{itemize}
\item edellisen perusteella tiedetään, \textbf{miten} lambdalausekkeilla lasketaan, mutta vielä ei ole käytössä mitään \textbf{millä} laskea
\pause
\item Churchin numeraalit on eräs tapa muodostaa luonnolliset luvut lambdalausekkeilla (vertaa Peanon lukuihin)	 
\end{itemize}
\pause

\[ \textbf{0} \equiv \lambda fx . x \]
\[ \textbf{1} \equiv \lambda fx . fx \]
\[ \textbf{2} \equiv \lambda fx . f (fx) \] 
\[ \textbf{3} \equiv \lambda fx . f(f(fx)) \]
\[ \vdots \]
\[ \textbf{N} \equiv \lambda fx . \underbrace{ f ( f \ldots (f }_{ N-\text{kappaletta}} x)) \ldots ) \] 
\end{frame}

\begin{frame}
\frametitle{Churchin numeraalit - Aritmetiikkaa}
\begin{itemize}
\item Churchin numeraalit rakentuvat siten, että niillä voidaan (kohtalaisen) helposti määritellä esimerkiksi yhteen- ja kertolaskufunktiot \pause \alert{... tietenkin lambdalausekkeita käyttäen!}
\end{itemize}
\pause
\[ \textbf{ADD} \equiv \lambda nmfx . n f (m f x) \] 
\pause
\[ \textbf{MUL} \equiv \lambda n m f . n (mf) \]
\end{frame}

\begin{frame}
\frametitle{Churchin numeraalit - yhteenlasku}
\begin{align*}
\textbf{ADD 2 2} \\
\onslide<2->{ &\equiv (\lambda nmfx . n f (m f x)) \; (\lambda fx . f (fx))  \; (\lambda fx . f (fx)) \\}
\onslide<3->{ &\rightarrow_{\beta} (\lambda mfx . (\lambda fx . f (fx)) \; f  \; (m f x)) \; (\lambda fx . f (fx)) \\}
\onslide<4->{ &\rightarrow_{\beta} (\lambda fx . (\lambda fx . f (fx)) \; f  \; (\lambda fx . f (fx)) \; f \; x)) \\}
\onslide<5->{ &\rightarrow_{\beta} (\lambda fx . (\lambda x . f (fx)) \; (\lambda fx . f (fx)) \; f \; x)) \\}
\onslide<6->{ &\rightarrow_{\beta} (\lambda fx . f (f(\lambda fx . f (fx)) \; f \; x))) \\}
\onslide<7->{ &\rightarrow_{\beta} (\lambda fx . f (f(\lambda x . f (fx)) \; x))) \\}
\onslide<8->{ &\rightarrow_{\beta} (\lambda fx . \underbrace{f (f(f (f}_{4-kappaletta} x)))) \\}
\onslide<9>{ &\equiv \textbf{4} }
\end{align*}
\end{frame}

\begin{frame}
\frametitle{Churchin numeraalit - kertolasku}
\begin{align*}
\textbf{MUL 2 3} \\
\onslide<2->{ &\equiv (\lambda n m f . n (mf)) \; (\lambda fx . f (fx))  \; (\lambda fx . f (f(f(x))) \\}
\onslide<3->{ &\rightarrow_{\beta} (\lambda mf . (\lambda fx . f (fx)) \; (m f)) \; (\lambda fx . f (f(fx))) \\}
\onslide<4->{ &\rightarrow_{\beta} (\lambda f . (\lambda fx . f (fx)) \; ((\lambda fx . f (f(fx))) f)) \\}
\onslide<5->{ &\rightarrow_{\beta} (\lambda f . (\lambda x . ((\lambda fx . f (f(fx))) f)) (((\lambda fx . f (f(fx))) f))x)) \\}
\onslide<6->{ &\rightarrow_{\beta} (\lambda f . (\lambda x . (\lambda x . f (f(fx)))) (((\lambda fx . f (f(fx))) f))x)) \\}
\onslide<7->{ &\rightarrow_{\beta} (\lambda f . (\lambda x . f (f(f((\lambda fx . f (f(fx))) f))x)))) \\}
\onslide<8->{ &\rightarrow_{\beta} (\lambda f . (\lambda x . f (f(f(\lambda x . f (f(fx))))x))))) \\}
\onslide<8->{ &\rightarrow_{\beta} (\lambda f . (\lambda x . \underbrace{ f (f(f(f(f(f }_{6-kappaletta} x))))))) \\}
\onslide<9>{ &\equiv \textbf{6} }
\end{align*}
\end{frame}


\begin{frame}
\frametitle{Churchin numeraalit - huomautus}
\begin{itemize}

\item Churchin numeraalit eivät ole erityisen tehokas apuväline yhteen- ja kertolaskujen suorittamiseen

\pause
\par
\item Luvun $n$ esittäminen Churchin numeraalina vie tilaa \textbf{O(n)}
\pause
\par
\item Churchin numeraalien merkitys on lähinnä teoreettinen: lambdakalkyyli on tarpeeksi ilmaisuvoimainen laskennan malliksi

\end{itemize}
\end{frame}


\begin{frame}
\begin{itemize}
\frametitle{Controllin ohjaus}

\item If-Then-Else -rakenne voidaan toteuttaa varsin helposti lambdalausekkeilla
\par
\pause

\item $TRUE \equiv \lambda xy . x$
\pause
\item $FALSE \equiv \lambda xy . y$
\pause
\item $IfThenElse \equiv \lambda pxy . pxy$
\pause
\par

\item Myös rekursiota voidaan simuloida lambdalausekkeilla:
\pause
\par
\item On olemassa lambdalauseke \textbf{Y}, jolle pätee $\textbf{Y} \; f \rightarrow_{\beta} \; f \; (\textbf{Y} \; f)$
\end{itemize}
\end{frame}



\begin{frame}
\begin{itemize}
\frametitle{Funktionaalinen ohjelmointi ja lambdakalkyyli}

\item Monet funktionaalisen ohjelmoinnin tekniikat ovat peräisin lambdakalkyylista:
\par
\pause
\item Anonyymit funktiot
\par
\pause
\item Curryn muunnos ja osittain sovelletut funktiot
\par
\pause
\item Funktioargumentit ja funktiopaluuarvot

\end{itemize}
\end{frame}




%\begin{frame}
%\frametitle{Sovellus - LISP}
%\begin{itemize}
%\item LISP-ohjelmointikieli voidaan nähdä lambdakalkyylin implementaationa, jota on %täydennetty vakiolla
%\end{frame}

%\end{itemize}
\end{document}
