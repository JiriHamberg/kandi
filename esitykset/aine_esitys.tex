\documentclass[12pt]{beamer}

\usepackage[finnish]{babel}
\usepackage[utf8]{inputenc}


\begin{document}


\title{Lambdakalkyyli}
\author{Jiri Hamberg}
\titlegraphic{\includegraphics[scale=0.4]{xkcd-functional.png}}
\frame{\maketitle}

\begin{frame}
\frametitle{Historiaa}

\begin{itemize}
\item loogikko David Hilbert esitti vuonna 1928 nimellä Entscheidungsproblem tunnetun haasteen:
	\begin{itemize}
	\item anna algoritmi, joka ottaa syötteenään joukon aksioomia ja lauseen ja kertoo voidaanko lause päätellä annetuista aksioomista
	\item \alert{ongelma: miten määritellä algoritmi ja laskettavuus?}
\pause	
	\end{itemize}
\item vuonna 1936 Church ja Turing esittivät omat ratkaisuehdotuksensa Hilbertin haasteeseen
	\begin{itemize}
	\item Alonzo Church: lambdakalkyyli	
	\item Alan Turing: Turingin kone
	\end{itemize}
\pause
\item Turing myös todisti että lambdakalkyyli $\Leftrightarrow$ Turingin koneet  
	\begin{itemize} \item \alert{Churchin-Turingin teesi} \end{itemize} 
\end{itemize}
\end{frame}

\begin{frame}
\frametitle{Lambdalausekkeet}
Tavoite:
\[ 
(\lambda x \;. \; 3x + 5) \; 6 \; \rightarrow \; 3 * 6 + 5 \; \rightarrow \; 18 + 5 \; \rightarrow \; 23 
\]
\pause
\[ 
(\lambda x \;. \; 2x) \; (\lambda x \;. \; x + 3) \; \rightarrow \; \lambda x \; . \; 2(x + 3) 
\]
\end{frame}

\begin{frame}
\frametitle{Lambdalausekkeet}
Määritelmä: \\
Lambdalausekkeiden joukko $E$:

\[ V \subset E \textbf{  (muuttuja)}\]
\[ \text{Jos } e_{1} \in E \text{ ja } e_{2} \in E, \text{ niin }  (e_{1}e_{2}) \in E \textbf{  (sovellus)} \]
\[ \text{Jos } x \in V \text{ ja } e \in E, \text{ niin } \lambda x.e \in E \textbf{  (abstraktio)}\]

missä $V$ on numeroituva symbolijoukko, esimerkiksi ASCII-merkkijonojen joukko.

\pause
\par
Esimerkkejä:
\[ ((xy)z)  \Leftrightarrow xyz \textbf{  (Notaatio)} \]
\[ \lambda x . xyz \]
\[ (\lambda x . (\lambda y . xyx)) z \Leftrightarrow \lambda xy . xyz \textbf{  (Notaatio, Curryn muunnon) } \]
\par
\pause
\alert{Määritelmä $\rightarrow$ Tavoite ???}
\end{frame}

\begin{frame}
\frametitle{Reduktiot ja $\alpha$-konversio}
\begin{itemize}
\item lambdalausekkeita voidaan "sieventää" käyttäen $\beta$- ja $\eta$-reduktiota sekä $\alpha$-konversiota
\pause
\item $\alpha$-konversio uudelleennimeää lausekkeen sidottuja muuttujia:
\[ \lambda x . xyz  \rightarrow_{\alpha} \lambda w . wyz  \]
\pause
\item $\beta$-reduktio "sijoittaa" sovelletun lausekkeen lambda-abstraktion muuttujaan:
\[ (\lambda xy . xyx) z  \rightarrow_{\beta} \lambda y . zyz \]
\pause
\item $\eta$-reduktio poistaa "turhan" lambda-abstraktion:
\[ \lambda xy . (yzy) x \rightarrow_{\eta} \lambda y . yzy \]
\end{itemize}
\end{frame}

\begin{frame}
\frametitle{Semantiikkaa}
\begin{itemize}
\item edellisen perusteella tiedetään, \textbf{miten} lambdalausekkeilla lasketaan, mutta vielä ei ole käytössä mitään \textbf{millä} laskea
\pause
\item Churchin numeraalit on eräs tapa muodostaa luonnolliset luvut lambdalausekkeilla (vertaa Peanon lukuihin)	 
\end{itemize}
\pause

\[ \textbf{0} \equiv \lambda fx . x \]
\[ \textbf{1} \equiv \lambda fx . fx \]
\[ \textbf{2} \equiv \lambda fx . f (fx) \] 
\[ \textbf{3} \equiv \lambda fx . f(f(fx)) \]
\[ \vdots \]
\[ \textbf{N} \equiv \lambda fx . \underbrace{ f ( f \ldots (f }_{ N-\text{kappaletta}} x)) \ldots ) \] 
\end{frame}

\begin{frame}

\end{frame}

\end{document}