% --- Template for thesis / report with tktltiki2 class ---
% 
% last updated 2013/02/15 for tkltiki2 v1.02

\documentclass[finnish, grading]{tktltiki2}
\documentclass[12pt]{article}

% tktltiki2 automatically loads babel, so you can simply
% give the language parameter (e.g. finnish, swedish, english, british) as
% a parameter for the class: \documentclass[finnish]{tktltiki2}.
% The information on title and abstract is generated automatically depending on
% the language, see below if you need to change any of these manually.
% 
% Class options:
% - grading                 -- Print labels for grading information on the front page.
% - disablelastpagecounter  -- Disables the automatic generation of page number information
%                              in the abstract. See also \numberofpagesinformation{} command below.
%
% The class also respects the following options of article class:
%   10pt, 11pt, 12pt, final, draft, oneside, twoside,
%   openright, openany, onecolumn, twocolumn, leqno, fleqn
%
% The default font size is 11pt. The paper size used is A4, other sizes are not supported.
%
% rubber: module pdftex

% --- General packages ---
\usepackage[utf8]{inputenc}
\usepackage[T1]{fontenc}
\usepackage{lmodern}
\usepackage{microtype}
\usepackage{amsfonts,amsmath,amssymb,amsthm,booktabs,color,enumitem,graphicx}
\usepackage[pdftex,hidelinks]{hyperref}
\usepackage{cancel}

% Automatically set the PDF metadata fields
\makeatletter
\AtBeginDocument{\hypersetup{pdftitle = {\@title}, pdfauthor = {\@author}}}
\makeatother

% --- Language-related settings ---
%
% these should be modified according to your language

% babelbib for non-english bibliography using bibtex
%\usepackage[fixlanguage]{babelbib}
\usepackage[finnish]{babelbib}
\selectbiblanguage{finnish}

% add bibliography to the table of contents
\usepackage[nottoc]{tocbibind}
% tocbibind renames the bibliography, use the following to change it back
\settocbibname{Lähteet}

% --- Theorem environment definitions ---

\newtheorem{lau}{Lause}
\newtheorem{lem}[lau]{Lemma}
\newtheorem{kor}[lau]{Korollaari}

\theoremstyle{definition}
\newtheorem{maar}[lau]{Määritelmä}
\newtheorem{ong}{Ongelma}
\newtheorem{alg}[lau]{Algoritmi}
\newtheorem{esim}[lau]{Esimerkki}

\theoremstyle{remark}
\newtheorem*{huom}{Huomautus}


% --- tktltiki2 options ---
%
% The following commands define the information used to generate title and
% abstract pages. The following entries should be always specified:

\title{Lambdakalkyyli}
\author{Jiri Hamberg}
\date{\today}
%\level{Seminaariraportti}
%\level{Kandidaatintutkielma}

\abstract{Tarkastellaan lambdalausekkeita matemaattisen logiikan formaalina järjestelmänä sekä laskennan mallina. Määritellään lausekkeille reduktiosäännöt ja ekvivalenssirelaatiot ja esitellään Churchin-Rosserin teoreemat. Lopuksi näytetään miten edellä kuvatut käsitteet soveltuvat kokonaislukuaritmetiikan mallintamiseen.}

% The following can be used to specify keywords and classification of the paper:

\keywords{lambda calculus }

% classification according to ACM Computing Classification System (http://www.acm.org/about/class/)
% This is probably mostly relevant for computer scientists
% uncomment the following; contents of \classification will be printed under the abstract with a title
% "ACM Computing Classification System (CCS):"
% \classification{}

% If the automatic page number counting is not working as desired in your case,
% uncomment the following to manually set the number of pages displayed in the abstract page:
%
% \numberofpagesinformation{16 sivua + 10 sivua liitteissä}
%
% If you are not a computer scientist, you will want to uncomment the following by hand and specify
% your department, faculty and subject by hand:
%
% \faculty{Matemaattis-luonnontieteellinen}
\department{Tietojenkäsittelytieteen laitos}
% \subject{Tietojenkäsittelytiede}
%
% If you are not from the University of Helsinki, then you will most likely want to set these also:
%
\university{Helsingin Yliopisto}
% \universitylong{HELSINGIN YLIOPISTO --- HELSINGFORS UNIVERSITET --- UNIVERSITY OF HELSINKI} % displayed on the top of the abstract page
\city{Helsinki}
%

\renewcommand{\baselinestretch}{1.5}

\begin{document}

\setlength{\parskip}{0.3cm}
% --- Front matter ---

\frontmatter      % roman page numbering for front matter

\maketitle        % title page
\makeabstract     % abstract page

\tableofcontents  % table of contents

% --- Main matter ---

\mainmatter       % clear page, start arabic page numbering

%\section{Esimerkkiluku}

% Write some science here.

%Esimerkkilause ja lähdeviite~\cite{esimerkki}.
\section{Johdanto}

Imperatiivisissa ohjelmoinnissa ohjelma esitetään sarjana komentoja, jotka muokkaavat ohjelman tilaa (\textit{state}). Ohjelman tilalla tarkoitetaan ohjelman muuttuja-arvo -parien kokoelmaa. Imperatiivisen ohjelmointiparadigman teoreettisena perustana on laskennan malli, joka tunnetaan Turingin koneena. Turingin kone koostuu nauhasta ja lukupäästä sekä tilasiirtymäfunktiosta. Nauha ja lukupää muodostavat yhdessä Turingin koneen tilan, joka vastaa imperatiivisen ohjelman tilaa. Siirtymäfunktio puolestaan vastaa imperatiivista ohjelmakoodia. Turingin kone esittää laskennan sarjana nauhan ja lukupään tilanmuutoksia, jotka siirtymäfunktio määrittelee. Juuri laskennan esittäminen sarjana tilanmuutoksia karakterisoi parhaiten imperatiivisen ohjelmointiparadigman~\cite[s.~3]{Hudak89}.
\par
Deklaratiivisessa ohjelmoinnissa ohjelman esitys on yksi tai useampi lauseke. Ohjelman suorittaminen merkitsee lausekkeiden arvojen laskemista. Huomattavin ero deklaratiivisen ja imperatiivisen paradigman välillä on se, että deklaratiiviset ohjelmat ovat tilattomia. Eräs tilattomuuden suurista eduista suhteessa imperatiiviseen ohjelmointityyliin on viittausläpinäkyvyys (\textit{referential transparency}). Lausekkeen viittausläpinäkyvyys tarkoittaa sitä, että lausekkeen mikä tahansa alilauseke voidaan korvata toisella lausekkeella, jonka arvo on sama kuin alilausekkeen arvo, muuttamatta ohjelman semantiikkaa~\cite[s.~5]{Hudak89}..  
\par
Funktionaalisen ohjelmoinnin teoreettisena perustana on Alonzo Churchin 1930-luvulla kehittämä lambdakalkyyli ~\cite[s.~37--50]{PJ1987}. Lambdakalkyyli on laskennan malli, jossa laskennalla tarkoitetaan annetun lambdalausekkeen arvon laskemista käyttäen lambdalausekkeille määriteltyjä reduktiosääntöjä.
\section{Lambdalausekkeet ja vapaat muuttujat}

\subsection{Lambdalausekkeet}

Lambdalausekkeet koostuvat muuttujasymboleista, lambda-abstraktioista ja lambdalausekkeiden sovelluksista muille lambdalausekkeille.
\par 
Muuttujasymbolit ovat lamdalausekkeiden terminaalisymboleita, eräänlaisia lausekkeiden perusosia. Muuttujasymboleiden joukon valinnalla ei ole teorian kannalta merkitystä, kunhan symboleita on äärettömän monta ja ne voidaan erottaa toisistaan. Usein käytetään esimerkiksi ASCII-merkistön merkkijonoja.
\par
Lambda-abstraktioiden ideana on toimia yksiparametristen funktioiden syntaktisena mallina. Esimerkiksi identiteettifunktio voidaan ilmaista lambdalausekkeella: 
\[ \underbrace{ \lambda x }_{ f(x) } \underbrace{ . }_{ = }  \underbrace{ x }_{x} \]
Lambda-abstraktio koostuu muuttujasta ja rungosta. Muuttuja on lambda-merkin ja pisteen väliin jäävä symboli ja runko on pisteen oikealla puolella oleva lambdalauseke. 
\par
Sovellus pyrkii mallintamaan funktioon tehtyä sijoitusoperaatiota. Muuttujasymbolin $y$ sovellus ylläesitetylle identiteettifunktiolle voitaisiin ilmaista lambdalausekkeena seuraavasti:
\[ ( \underbrace{ (\lambda x . x ) }_{ \text{kohde} } \; \underbrace{ y }_{ \text{sovellettava lauseke} } ) \]
\par
Formaalisti lambdalausekkeiden syntaksi voidaan muotoilla seuraavasti~\cite[s.~8]{Hudak89}:
\pagebreak
\begin{maar}[lambdalausekkeet]
Lambdalausekkeiden joukko $E$ määritellään rekursiivisesti: 
\[ V \subset E \]
\[ \text{Jos } e_{1} \in E \text{ ja } e_{2} \in E, \text{ niin }  (e_{1} \; e_{2}) \in E \]
\[ \text{Jos } x \in V \text{ ja } e \in E, \text{ niin } \lambda x.e \in E \]

missä $V$ on ääretön joukko muuttujia. Muuttujia on tapana merkitä symboleilla $x, y,z...$ tai vaihtoehtoisesti $x_{1}, x_{2}, x_{3}...$ ja mielivaltaisia lambdalausekkeita puolestaan symboleilla $L, N, M ...$ tai vaihtoehtoisesti $e_{1}, e_{2}, e_{3}...$.  Lauseketta, joka on muotoa $\lambda x.e$, kutsutaan lambda-abstraktioksi tai lyhyesti abstraktioksi. Lauseketta, joka on muotoa $(e_{1} \; e_{2})$, kutsutaan lausekkeen $e_{2}$ sovellukseksi $e_{1}$:lle tai lyhyesti sovellukseksi.
\end{maar}

Sovelluksille on tapana käyttää vasemmalta oikealle assosioivaa lyhennysmerkintää: 
\[e_{1} e_{2} e_{3} \equiv ((e_{1} \; e_{2}) \; e_{3})\]
Abstraktioille puolestaan käytetään seuraavaa oikealta vasemmalle assosioivaa lyhennysmerkintää, joka tunnetaan nimellä Curryn muunnos: 
\[ \lambda x_{1}x_{2}...x_{n}.e \equiv (\lambda x_{1} . ( \lambda x_{2} . ( \: ... \: ( \lambda x_{n} . e ))) \: ... \: ) \]

\par
Esimerkiksi seuraavat ovat lambdalausekkeita:
%\pagebreak
\[ x \]
\[ xyz \]
\[ \lambda x . xyz \]
\[ (\lambda xy . xyx) z \]

\par

Kukin lambda-abstraktio sisältää täsmälleen yhden muuttujan, joten voi vaikuttaa siltä, että lambdalausekkeilla voidaan mallintaa vain yksiparametrisia funktioita. Tämä näennäinen ongelma ratkeaa varsin helposti, sillä lambda-abstraktion runko voi sisältää lambda-abstraktion. Esimerkiksi:
\[ \lambda \underbrace{x}_{\text{muuttuja 1}} . ( \lambda \underbrace{y}_{\text{muuttuja 2}} . \underbrace{(x \; y)}_{\text{funktion runko}} ) \]
Käyttämällä Curryn muunnosta edellisen lausekkeen yhteys kaksiparametriseen funktioon nähdään vieläkin helpommin:
\[ \lambda \underbrace{xy}_{\text{f(x,y)}} \underbrace{.}_{=} \underbrace{xy}_{\text{funktion runko}}  \]
%Lambdalausekkeita reduktiosääntöjen kannalta tärkeitä käsitteitä ovat lambdalausekkeiden vapaat ja sidotut muuttujat sekä %substituutiosäännöt. Lambdalausekkeiden vapaat muuttujat ovat intuitiivisesti hyvin samanlaisia kuin ohjelmointikielissä yleensäkin: %muuttujien sidonnat vaikuttavat lausekehierarkiassa ylhäältä alaspäin ja hierarkiassa alempana sijaitsevat sidonnat sitovat vahvemmin kuin %hierarkiassa ylempänä sijaitsevat sidonnat.   

\subsection{Vapaat muuttujat}

Lambdalausekkeen vapailla muuttujilla tarkoitetaan sellaisia lausekkeessa esiintyviä muuttujia, jotka eivät esiinny lausekkeessa lambda-abstraktion muuttujaosana~\cite[s.~8]{Hudak89}. Muuttujien jaottelu vapaisiin ja sidottuihin on olennaista myöhemmin määriteltävien reduktiosääntöjen kannalta. Lambdalausekkeiden vapaat ja sidotut muuttujat ovat käsitteellisesti hyvin läheistä sukua aliohjelmien vapaille ja sidotuille muuttujille, sillä lambda-abstraktion muuttujaosa voidaan perustellusti samastaa aliohjelman argumenttilistaan.

\begin{maar}[Vapaat ja sidotut muuttujat]
Lambdalausekkeen $e$ vapaat muuttujat, joita merkitään $fv(e)$, ovat:
\begin{align*} 
fv(x) &= \{x\}\ \text{ jos } x \text{ on muuttuja} \\
fv(e_{1}e_{2}) &= fv(e_{1}) \cup fv(e_{2}) \\
fv(\lambda x.e) &= fv(e) - \{x\}
\end{align*}
Muuttuja $x$ on vapaa lambdalausekkeessa $e$, jos $x \in fv(e)$. Muuten $x$ on sidottu lambdalausekkeessa $e$.
\end{maar} 

\subsection{Substituutiosäännöt}
Jotta lambdalausekkeiden reduktiosäännöt voidaan määritellä, tarvitaan avuksi vielä seuraavat substituutiosäännöt ~\cite[s.~8]{Hudak89}. Substituutiosäännöt kuvaavat nimensä mukaisesti sitä, miten lambdalausekkeiden muuttujia voidaan korvata muilla lambdalausekkeilla, kun lambda-abstraktioon kohdistuu sovellus.
\par
Tarkastellaan seuraavanlaista sovelluslauseketta:
\[ (\lambda yx . yx) x \]
Tavoitteena on määritellä substituutiosäännöt siten, että sovelluksen kohteena olevan lausekkeen muuttujaosan ilmentymät korvataan sovellettavalla lausekkeella. Tarkastellaan mitä ongelmia liittyy yksinkertaiseen tekstisubstituutioon yllä olevan lausekkeen tapauksessa.
\[  
	(\lambda yx . y \; x) x  \rightarrow_{\text{tekstisubstituutio}} \label{eq:1} \tag{\textbf{1}}
	\lambda x . x \; x
\]
Substituutiosäännöt haluttaisiin kuitenkin määritellä siten, että lambda-abstraktion muuttujaosan nimellä ei ole merkitystä substituution lopputuloksen kannalta. Jos kohdassa \eqref{eq:1} sovelluksen kohteena olevan lausekkeen muuttujaosa olisi nimetty toisin, esimerkiksi:
\[  (\lambda yz . y \; z)  \]
niin tällöin tekstisubstituutio olisi johtanut rakenteeltaan varsin erityyppiseen lausekkeeseen:
\[  
	(\lambda yz . y \; z) x  \rightarrow_{\text{tekstisubstituutio}} \label{eq:2} \tag{\textbf{2}}
	\lambda z . x \; z
\]
Tapauksessa \eqref{eq:1} lausekkeella ei substituution jälkeen ole vapaita muuttujia, mutta tapauksessa \eqref{eq:2} lausekkeella on substituution jälkeen vapaa muuttuja $x$.
Näiden naiiviin tekstisubstituutioon liittyvien ongelmien ratkaisemiseksi määritellään seuraavat substituutiosäännöt, joiden tarkoituksena on ratkaista vapaiden ja sidottujen muuttujien yhteentörmäykseen liittyvät ongelmat substituutioissa.

\begin{maar}[substituutiosäännöt]
Olkoon $e$ lambdalauseke ja $x$ muuttuja. Muuttujan $x$ substituutiota lambdalausekkeella $e$ lambdalausekkeessa $e_{1}$ merkitään $[e/x] e_{1}$ ja se määritellään rekursiivisesti lausekkeen $e_{1}$ rakenteen suhteen:  

%\[[e/x]x_{1} = 
%        \begin{cases}
%                e & \text{jos } x = x_{1} \\
%                x_{1} & \text{muulloin}
%        \end{cases}
%        \label{eq:S1} \tag{\textbf{S1}}
%\]
%\[ [e/x](e_{1} \; e_{2}) = ([e/x]e_{1} \; [e/x]e_{2}) \label{eq:S2} \tag{\textbf{S2}}\]
%\[[e/x](\lambda x_{1}.e_{1}) = 
%        \begin{cases}
%                \lambda x_{1}.e_{1} & \text{jos } x = x_{1} \\
%                \lambda x_{1}.[e/x]e_{1} & \text{ jos } x \neq x_{1} \text{ ja } x_{1} \notin fv(e) \\
%                \lambda x_{i}.[e/x]([x_{i}/x_{1}]e_{1}) & \text{muulloin, missä } x_{i} \neq x \text { ja } %x_{i} \neq x_{1} \text{ ja } x_{i} \notin fv(e) \cup fv(e_{1})
%        \end{cases}
% \label{eq:S3} \tag{\textbf{S3}}       
%\]

\begin{align*}
[e/x]x_{1} &= 
        \begin{cases}
                e & \text{jos } x = x_{1} \\
                x_{1} & \text{muulloin}
        \end{cases}
        \label{eq:S1} \tag{\textbf{S1}} \\
[e/x](e_{1} \; e_{2}) &= ([e/x]e_{1} \; [e/x]e_{2}) \label{eq:S2} \tag{\textbf{S2}} \\
[e/x](\lambda x_{1}.e_{1}) &= 
        \begin{cases}
                \lambda x_{1}.e_{1} & \text{jos } x = x_{1} \label{eq:S3} \tag{\textbf{S3}} \\
                \lambda x_{1}.[e/x]e_{1} & \text{ jos } x \neq x_{1} \text{ ja } x_{1} \notin fv(e) \\
                \lambda x_{i}.[e/x]([x_{i}/x_{1}]e_{1}) & \text{muulloin, missä } x_{i} \neq x \text { ja } x_{i} \neq x_{1} \text{ ja } x_{i} \notin fv(e) \cup fv(e_{1})
        \end{cases}       
\end{align*}

\end{maar} 

Määritelmän olennaisin sisältö on se, että substituution kohteena olevan lambdalausekkeen sidotut muuttujat tulee uudelleennimetä ennen sijoituksen tekemistä siinä tapauksessa, että substituution suorittaminen muuttaisi substituution kohteena olevan lausekkeen tai sijoitettavan lausekkeen vapaita muuttujia sidotuiksi. Esimerkiksi edellä käsiteltyyn sovellukseen $(\lambda yx . y \; x) x$ liittyvät ongelmat ratkeavat seuraavalla tavalla substituutiosääntöjen avulla:

%\[ [\lambda x . x / y]((\lambda x . yx)  \stackrel{\eqref{eq:S3}}{=} \\ 
%	\lambda z . [\lambda x . x / y](yz) \stackrel{\eqref{eq:S2}}{=} \\
%	\lambda z . [\lambda x . x / y]y [\lambda x . x / y]z \stackrel{\eqref{eq:S1}}{=} \lambda z . (\lambda x . %x) z	
%\]
\[ [ x / y]((\lambda x . y \; x)  \stackrel{\eqref{eq:S3}}{=} \\ 
	\lambda z . [ x / y](y \; z) \stackrel{\eqref{eq:S2}}{=} \\
	\lambda z . [ x / y]y [ x / y]z \stackrel{\eqref{eq:S1}}{=} \lambda z . x \; z	
\]   


\section{Reduktiosäännöt ja yhtäsuuruus}

\subsection{Reduktiosäännöt}
Lambdalausekkeiden kiinnostavimpia piirteitä lienee, että niille voidaan muotoilla varsin yksinkertaiset laskusäännöt, jotka kuitenkin riittävät tekemään lambdakalkyylista Turing-täydellisen~\cite[liite]{Turing36}.  
\par
Kuten jo edellä mainittiin, voidaan lambda-abstraktio samastaa anonyymin funktion kanssa. On luontevaa ajatella, että lambdalausekkeen $M$ sovellus lambda-abstraktiolle $\lambda x . N$ vastaa abstraktin funktion $\lambda x . N$ arvoa argumentilla $M$. Tällöin reduktiosäännöt voidaan ymmärtää näiden anonyymien funktioden sievennyssäännöiksi.
\par
Määritellään seuraavaksi lambdalausekkeiden reduktiosäännöt~\cite[s.~9]{Hudak89}. Reduktiosäännöt kertovat miten lambdalausekkeita voidaan sieventää muuttamatta niiden arvoa. Beta-reduktio ilmaisee miten lambda-abstraktiosta ja mielivaltaisesta lambdalausekkeesta koostuvaa sovellusta voidaan sieventää. Eta-reduktiota puolestaan ilmaisee sen, että jos lambda-abstraktio ei tee muuta kuin soveltaa muuttujaansa lambdalausekkeeseen $e$, niin abstraktion arvo on sama kuin lausekkeen $e$.    

\begin{maar}[reduktiosäännöt]	
	%\begin{enumerate}		
\[\beta \text{-reduktio: } \; (\lambda x.e )\; e_{1} \rightarrow_{\beta} [e_{1} / x]e \]
\[\eta \text{-reduktios: } \; \lambda x.(e \; x) \rightarrow_{\eta} e \text{ jos } x \notin fv(e) \]	
	%\end{enumerate}
Jos $\beta$-reduktion toistuva soveltaminen lausekkeen $N$ alilausekkeisiin tuottaa lausekkeen $M$, niin sanotaan, että $M$ on lausekkeen $N$ \textbf{beta-reduktio} ja merkitään $N  \twoheadrightarrow_{\beta} M$. Vastaavasti Jos $\eta$-reduktion toistuva soveltaminen lausekkeen $N$ alilausekkeisiin tuottaa lausekkeen $M$, niin sanotaan, että $M$ on lausekkeen $N$ \textbf{eta-reduktio} ja merkitään $N  \twoheadrightarrow_{\eta} M$. Lisäksi, jos lauseke $M$ saadaan lausekkeesta $N$ soveltamalla sen alilausekkeisiin toistuvasti eta- tai beta-reduktiota, niin sanotaan, että $M$ on lausekkeen $N$ \textbf{reduktio} ja merkitään $N  \twoheadrightarrow M$        
\end{maar}

%\begin{esim}[$\beta$-reduktio]
%<ESIMERKKI>
%\end{esim}
%
%\par

\subsection{Lausekkeiden yhtäsuuruus}

Seuraavaksi esiteltävät relaatiot~\cite[s.~23--24]{HBEB2000} karakterisoivat lambdalausekkeiden arvojen yhtäsuuruuden ja niiden ideana on toimia yksinkertaisina sääntöinä, joiden avulla voidaan päätellä ovatko kaksi lambdalauseketta arvoiltaan yhtäsuuret.
\par
Intuitiivisesti on selvää, että jos kaksi lauseketta voidaan sieventää samaan muotoon, niin lausekkeilla on sama arvo. Lausekkeiden yhtäsuuruus määritelläänkin pitkälti edellä annettujen sievennyssääntöjen avulla. Lisäksi halutaan, että kaksi lambdalauseketta ovat yhtä suuret, jos ne eroavat toisistaan vain sidottujen muuttujiensa nimien suhteen.  
\par
Kaksi lambdalauseketta ovat alfa-ekvivalentit, mikäli ne ne eroavat toisistaan ainoastaan sidottujen muuttijiensa nimien suhteen. Kaksi lauseketta ovat beta- tai eta-ekvivalentit jos toinen niistä saadaan toisesta käyttämällä beta- tai eta-reduktiota.

\begin{maar}[lambdalausekkeiden arvojen yhtäsuuruus]
	%\begin{enumerate}
		 
\[ 
	\alpha \text{-ekvivalenssi: } \; \lambda x.e =_{\alpha} \lambda x_{i}.		[x/x_{i}]e \text{,  } \forall x_{i} \notin fv(e) 
\]
\[ 
	\beta \text{-ekvivalenssi: } \; N =_{\beta} M \text{ jos ja vain jos } N 		\twoheadrightarrow_{\beta} M \text{ tai }
	M \twoheadrightarrow_{\beta} N 
\]
\[ 
	\eta \text{-ekvivalenssi: } \; N =_{\eta} M \text{ jos ja vain jos } N 		\twoheadrightarrow_{\eta} M \text{ tai }
	M \twoheadrightarrow_{\eta} N 
\]
	%\end{enumerate}
Lambdalausekkeet $N$ ja $M$ ovat arvoltaan yhtäsuuret, mikäli	on olemassa lambdalauseke $L$ siten että $N$ ja $L$ ovat keskenään alfa- beta- tai eta-ekvivalenttit ja $M$ ja $L$ ovat keskenään alfa-, beta- tai eta-ekvivalentit.	
\end{maar}

Intuitiivisesti edellinen määritelmä tarkoittaa, että kaksi lambdalauseketta ovat arvoiltaan yhtäsuuret, mikäli on olemassa lauseke, joka voidaan johtaa molemmista lausekkeista käyttäen eta- ja beta-reduktiota sekä uudelleennimeämällä sidottuja muuttujia. 

%\par

 %Esimerkiksi seuraavat lambdalausekkeet ovat arvoiltaan yhtäsuuret:
 
%Lambdalausekkeen sanotaan olevan normaalimuodossa, jos siihen ei voida soveltaa $\eta$- tai $\beta$-reduktioita. Voidaan osoittaa, että kaikille lambdalausekkeille, jotka voidaan redusoida normaalimuotoon, pätee, että normaalimuoto saavutetaan valitsemalla aina lausekkeen vasemmanpuolisin alilauseke, jota voidaan vielä redusoida ja soveltamalla siihen $\eta$- tai $\beta$-reduktiota ja toistamalla tätä menettylyä kunnes lauseke on normaalimuodossa. Lisäksi voidaan osoittaa että jos lauseke voidaan redusoida normaalimuotoon, niin lausekkeen normaalimuoto on yksikäsitteinen. Näinollen lambdalausekkeen arvon laskemisella tarkoitetaan lausekkeen redusoimista normaalimuotoon.
%\par
%Kaikille lambdalausekkeille ei ole olemassa normaalimuotoa. Alan Turing todisti vuonna 1937, että Turingin kone on laskennan mallina ekvivalentti lambdakalkyylin kanssa. Tämä tulos ei voisi pitää paikkaansa, jos jokaiselle lambdalausekkeelle olisi löydettävissä normaalimuoto. Muuten ratkeamattomaksi tunnettu pysähtymisongelma voitaisiin ratkaista simuloimalla Turingin koneita lambdalausekkeilla.
%\par 
%Vaikka lambdalausekkeet ovat rakenteeltaan melko yksinkertaisia, voidaan niiden avulla ymmärtää ja mallintaa monia funktionaalisen ohjelmoinnin kannalta tärkeitä käsitteitä. Koska lambda-abstraktiot voivat ottaa parametrinaan ja antaa paluuarvonaan lambdafunktion, ovat esimerkiksi korkeamman asteen funktiot varsin luonnollinen osa lambdakalkyylia.

\section{Yhteenveto}

Lambdalausekkeet muodostavat funktionaalisen ohjelmoinnin teoreettisen perustan. Lambdalausekkeilla voidaan varsin luonnollisesti esittää funktionaaliselle ohjelmoinnille tyypillisiä käsitteitä kuten Curryn-muunnos ja korkeamman kertaluvun funktiot. Lambdalausekkeiden laskenta on tilatonta ja sivuvaikutuksetonta. 

\par

Lambdalausekkeet voidaan esittää matemaattisen logiikan formaalina systeeminä. Lausekkeille voidaan määrittelä lekvivalenssirelaatiot sekä reduktiosäännöt käyttäen apuna vapaiden muuttujien ja substituutiosääntöjen käsitteitä. Reduktiosäännöt ja ekvivalenssirelaatiot voidaan ymmärtää lausekkeiden abstrakteina laskusääntöinä.

\par

Lambdalausekkeilla, joiden laskenta päättyy, on olemassa yksikäsitteinen normaalimuoto, joka voidaan samastaa lambdalausekkeen arvon kanssa. Algoritmi nimeltään normaalijärjestysevaluointi muuntaa lambdalausekkeen sen normaalimuotoon jos lausekkeella on normaalimuoto. Kokonaislukuja voidaan mallintaa lambdalausekkeiden avulla käyttäen Churchin numeraaleja. Churchin numeraaleille voidaan muodostaa esimerkiksi yhteen- ja kertolaskufunktiot käyttäen lambdalausekkeita. 

\par 

Tässä työssä ei ole käsitelty rekursiivisia lambdafunktioita, vaikka niihin liittyvät tulokset ovat sekä historiallisesti että lambdalausekkeiden sovellusten kannalta erittäin olennaisia. Muita kiinnostavia jatkokysymyksiä ovat lambdalausekkeiden sovellukset funktionaalisten kielten toteutuksissa, esimerkiksi LISP-kielessä; sekä tyypitetetyt lambdakalkyylit.

% --- References ---
%
% bibtex is used to generate the bibliography. The babplain style
% will generate numeric references (e.g. [1]) appropriate for theoretical
% computer science. If you need alphanumeric references (e.g [Tur90]), use
%
%\bibliographystyle{babalpha-lf}
%
% instead.

%\bibliographystyle{babplain-lf}
\bibliographystyle{babalpha-lf}
\bibliography{references-fi}


% --- Appendices ---

% uncomment the following

% \newpage
% \appendix
% 
% \section{Esimerkkiliite}

\end{document}
