% --- Template for thesis / report with tktltiki2 class ---
% 
% last updated 2013/02/15 for tkltiki2 v1.02

\documentclass[finnish]{tktltiki2}
\documentclass[12pt]{article}

% tktltiki2 automatically loads babel, so you can simply
% give the language parameter (e.g. finnish, swedish, english, british) as
% a parameter for the class: \documentclass[finnish]{tktltiki2}.
% The information on title and abstract is generated automatically depending on
% the language, see below if you need to change any of these manually.
% 
% Class options:
% - grading                 -- Print labels for grading information on the front page.
% - disablelastpagecounter  -- Disables the automatic generation of page number information
%                              in the abstract. See also \numberofpagesinformation{} command below.
%
% The class also respects the following options of article class:
%   10pt, 11pt, 12pt, final, draft, oneside, twoside,
%   openright, openany, onecolumn, twocolumn, leqno, fleqn
%
% The default font size is 11pt. The paper size used is A4, other sizes are not supported.
%
% rubber: module pdftex

% --- General packages ---

\usepackage[utf8]{inputenc}
\usepackage[T1]{fontenc}
\usepackage{lmodern}
\usepackage{microtype}
\usepackage{amsfonts,amsmath,amssymb,amsthm,booktabs,color,enumitem,graphicx}
\usepackage[pdftex,hidelinks]{hyperref}

% Automatically set the PDF metadata fields
\makeatletter
\AtBeginDocument{\hypersetup{pdftitle = {\@title}, pdfauthor = {\@author}}}
\makeatother

% --- Language-related settings ---
%
% these should be modified according to your language

% babelbib for non-english bibliography using bibtex
\usepackage[fixlanguage]{babelbib}
\selectbiblanguage{finnish}

% add bibliography to the table of contents
\usepackage[nottoc]{tocbibind}
% tocbibind renames the bibliography, use the following to change it back
\settocbibname{Lähteet}

% --- Theorem environment definitions ---

\newtheorem{lau}{Lause}
\newtheorem{lem}[lau]{Lemma}
\newtheorem{kor}[lau]{Korollaari}

\theoremstyle{definition}
\newtheorem{maar}[lau]{Määritelmä}
\newtheorem{ong}{Ongelma}
\newtheorem{alg}[lau]{Algoritmi}
\newtheorem{esim}[lau]{Esimerkki}

\theoremstyle{remark}
\newtheorem*{huom}{Huomautus}


% --- tktltiki2 options ---
%
% The following commands define the information used to generate title and
% abstract pages. The following entries should be always specified:

\title{Funktionaalinen ohjelmointi}
\author{Jiri Hamberg}
\date{\today}
%\level{Seminaariraportti}
%\level{Kandidaatintutkielma}

\abstract{Tiivistelmä.}

% The following can be used to specify keywords and classification of the paper:

\keywords{avainsana 1, avainsana 2, avainsana 3}

% classification according to ACM Computing Classification System (http://www.acm.org/about/class/)
% This is probably mostly relevant for computer scientists
% uncomment the following; contents of \classification will be printed under the abstract with a title
% "ACM Computing Classification System (CCS):"
% \classification{}

% If the automatic page number counting is not working as desired in your case,
% uncomment the following to manually set the number of pages displayed in the abstract page:
%
% \numberofpagesinformation{16 sivua + 10 sivua liitteissä}
%
% If you are not a computer scientist, you will want to uncomment the following by hand and specify
% your department, faculty and subject by hand:
%
% \faculty{Matemaattis-luonnontieteellinen}
% \department{Tietojenkäsittelytieteen laitos}
% \subject{Tietojenkäsittelytiede}
%
% If you are not from the University of Helsinki, then you will most likely want to set these also:
%
% \university{Helsingin Yliopisto}
% \universitylong{HELSINGIN YLIOPISTO --- HELSINGFORS UNIVERSITET --- UNIVERSITY OF HELSINKI} % displayed on the top of the abstract page
% \city{Helsinki}
%

\renewcommand{\baselinestretch}{1.5}

\begin{document}

\setlength{\parskip}{0.3cm}
% --- Front matter ---

\frontmatter      % roman page numbering for front matter

\maketitle        % title page
%\makeabstract     % abstract page

%\tableofcontents  % table of contents

% --- Main matter ---

\mainmatter       % clear page, start arabic page numbering

%\section{Esimerkkiluku}

% Write some science here.

%Esimerkkilause ja lähdeviite~\cite{esimerkki}.
Imperatiivisissa ohjelmoinnissa ohjelma esitetään sarjana komentoja, jotka muokkaavat ohjelman tilaa (\textit{state}). Ohjelman tilalla tarkoitetaan ohjelman muuttujien kokoelmaa. Imperatiivisen ohjelmointiparadigman teoreettisena perustana on laskennan malli, joka tunnetaan Turingin koneena. Turingin kone koostuu nauhasta ja lukupäästä sekä tilasiirtymäfunktiosta. Nauha ja lukupää muodostavat yhdessä Turingin koneen tilan, joka vastaa imperatiivisen ohjelman tilaa. Siirtymäfunktio puolestaan vastaa imperatiivista ohjelmakoodia. Turingin kone esittää laskennan sarjana nauhan ja lukupään tilanmuutoksia, jotka siirtymäfunktio määrittelee. Juuri laskennan esittäminen sarjana tilanmuutoksia karakterisoi parhaiten imperatiivisen ohjelmointiparadigman~\cite[p.~3]{Hudak89}.
\par
Deklaratiivisessa ohjelmoinnissa ohjelman esitys on joukko lausekkeita. Ohjelman suorittaminen merkitsee lausekkeiden arvojen laskemista. Huomattavin ero deklaratiivisen ja imperatiivisen paradigman välillä on se, että deklaratiiviset ohjelmat ovat tilattomia. Eräs tilattomuuden suurista eduista suhteessa imperatiiviseen ohjelmointityyliin on viittausläpinäkyvyys (\textit{referential transparency}). Lausekkeen viittausläpinäkyvyys tarkoittaa sitä, että lausekkeen mikä tahansa alilauseke voidaan korvata toisella lausekkeella, jonka arvo on sama kuin alilausekkeen arvo, muuttamatta ohjelman semantiikkaa~\cite[p.~5]{Hudak89}.
\par
Funktionaalinen ohjelmointiparadigma on osa deklaratiivista ohjelmointiparadigmaa. Funktionaalisessa ohjelmoinnissa evaluoitavat lausekkeet ovat funktioita. Imperatiivisessa ohjelmoinnissa sanaa funktio käytetään usein synonyyminä aliohjelmalle. Näin lavea määritelmä ei ole yhteensopiva funktion matemaattisen määritelmän kanssa, sillä matemaattisella funktiolla ei ole sivuvaikutuksia ja sen arvo riippuu yksikäsitteisesti funktion argumenteista. Funktionaalisen ohjelmoinnin viitekehyksessä matemaattisiin funktiohin rinnastettavissa olevista aliohjelmista käytetään termiä puhdas funktio (\textit{pure function}) erotuksena sivuvaikutuksia sisältäville tai paluuarvoiltaan monikäsitteisille aliohjelmille, joista vastaavasti käytetään termiä ei-puhdas funktio.  
\par
Funktionaalisen ohjelmoinnin teoreettisena perustana on Alonzo Churchin 1930-luvulla kehittämä lambdakalkyyli ~\cite[p.~37--50]{PJ1987}. Lambdakalkyyli on laskennan malli, jossa laskennalla tarkoitetaan annetun lambdalausekkeen arvon laskemista käyttäen lamdalausekkeille määriteltyjä reduktiosääntöjä. Lambdalauseke on joko muuttujasymboli, lambda-abstraktio tai lambdalausekkeen aplikaatio toiselle lambdalausekkeelle ~\cite[p.~9--36]{PJ1987}. Lambda-abstraktiot ovat anonyymejä yhden muuttujan funktiota. Lambda-abstraktio voi ottaa argumenttinaan toisen lambda-abstraktion ja myös lambda-abstraktion arvo voi olla lambda-abstraktio. Lambda-abstraktiot ovat siten yksinkertainen malli korkeamman asteen funktioille (\textit{higher order function}).
\par
Funktionaalisessa ohjelmoinnissa on toki kyse muustakin kuin ohjelman tilattomuuden ja viiteläpinäkyvyyden tavoittelusta. Korkeamman asteen funktiot, hahmontunnistus (\textit{pattern-matching}) ja laiska evaluaatio (\textit{lazy evaluation}) ovat eräitä tyypillisiä modernien funktionaalisten kielten ominaisuuksia~\cite{Hudak89}.  
\par
Ohjelmoinnin kannalta tärkeimpiä sivuvaikutuksia ovat I/O-operaatiot, jotka mahdollistavat informaation vaihtamisen ohjelman ja ulkomaailman välillä.
Koska puhtaat funktiot eivät voi aiheuttaa sivuvaikutuksia, on I/O-abstraktioiden toteuttaminen funktionaalisten kielten suunnittelun kannalta haasteellista. Kielissä kuten Scheme ja ML sivuvaikutuksia sisältävää koodia on mahdollista kirjoittaa, joskin sivuvaikutusten tarpeettoman runsasta käyttöä pidetään huonona ohjelmointityylinä~\cite[p.~23]{Hudak89}. Kielet kuten Haskell sen sijaan sisältävät tyyppijärjestelmiensä tasolla tuen monadiselle ohjelmoinnille, jossa sivuvaikutuksia tuottavat funktiot voidaan eristää puhtaista funktiosta kielen modulaarisuudesta tinkimättä~\cite[p.~4--16]{PJ2000}.       


% --- References ---
%
% bibtex is used to generate the bibliography. The babplain style
% will generate numeric references (e.g. [1]) appropriate for theoretical
% computer science. If you need alphanumeric references (e.g [Tur90]), use
%
%\bibliographystyle{babalpha-lf}
%
% instead.

%\bibliographystyle{babplain-lf}
\bibliographystyle{babalpha-lf}
\bibliography{references-fi}


% --- Appendices ---

% uncomment the following

% \newpage
% \appendix
% 
% \section{Esimerkkiliite}

\end{document}
